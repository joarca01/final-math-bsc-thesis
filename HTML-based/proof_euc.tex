\documentclass[a4paper, 12pt]{article}
\usepackage{geometry}
\geometry{top=2.54cm, bottom=2.54cm, left=2.54cm, right=2.54cm, heightrounded}
\usepackage[utf8]{inputenc}
\usepackage[T1]{fontenc}
\usepackage[english]{babel} 
\usepackage{amsthm, amsmath, amssymb, amsfonts, amscd}
\usepackage{mathtools}
\usepackage{fancyhdr}
\usepackage{setspace}
\usepackage{hyperref}
\usepackage{graphicx}
\usepackage{ifluatex}
\usepackage{array}
\usepackage{multirow}
\usepackage[table]{xcolor}
\usepackage{tcolorbox}
\usepackage{mdframed}

% ---- Spacing ---- %
\linespread{1.0}
\onehalfspacing

% ---- Page header ---- %
\pagestyle{fancy}
\headheight 30pt
\fancyhf{}
\rhead{Page \thepage}
\lhead{\nouppercase{Infinite number of primes in $5n+1$, $n\geqslant 0$}}
\headsep 1.5em

% ---- Link & URL style ---- %
\hypersetup{colorlinks, citecolor=green, linkcolor=blue, urlcolor=blue}
\urlstyle{rm}

% ---- Paragraph indentation style ---- %
\setlength{\parindent}{20pt}

% ---- Box ---- %
\definecolor{framecolor}{rgb}{0.5, 0.5, 0.5}
\definecolor{backcolor}{rgb}{0.94, 0.97, 1.0}
\newtcolorbox{mybox}{colback=backcolor, colframe=framecolor, boxrule=0.5pt, arc=3pt}

% ---- Theorems ---- %
\newtheorem{theorem}{Theorem}%[section]
\newtheorem{lemma}[theorem]{Lemma}

\theoremstyle{definition}
\newtheorem{definition}[theorem]{Definition}

% ---- Definitions ---- %
\renewcommand{\qedsymbol}{\rule{0.7em}{0.7em}} % Black box to end proofs. 
\newcommand{\Z}{\mathbb{Z}}
\newcommand{\Q}{\mathbb{Q}}
\newcommand{\C}{\mathbb{C}}
\newcommand{\F}{\mathbb{F}}
\DeclareMathOperator{\degpol}{deg} % Degree of a polynomial.
\DeclareUnicodeCharacter{03B6}{\zeta} % Define Unicode Character 'ζ'.
%\newcommand{\splitatcommas}[1]{%
%  \begingroup
%  \begingroup\lccode`~=`, \lowercase{\endgroup
%    \edef~{\mathchar\the\mathcode`, \penalty0 \noexpand\hspace{0pt plus 1em}}%
%  }\mathcode`,="8000 #1%
%  \endgroup
%}% Break long inline expressions after comma sign.
\renewcommand\title{\textbf{The arithmetic progression $5n+1$, $n\geqslant0$, contains infinitely many primes.\\ A Euclidean proof}}

\begin{document}
\thispagestyle{plain}
\pagenumbering{arabic}
\sloppy
%\begin{center}
%{\huge \bfseries \title\par}
%\vspace{1cm}
%\end{center}
\section{The arithmetic progression $5n+1$, $n\geqslant0$, contains infinitely many primes. A Euclidean proof}

We will prove that the arithmetic progression $\equiv 1 \pmod{5}$ contains infinitely many primes. Equivalently, we will see that there are infinitely many primes of the form $5n+1$, $n\geqslant0$. To follow the proof, one must recall the expression of the discriminant of a polynomial. 

\textbf{Definition.} The discriminant of a monic polynomial $A(x)=x^d+a_{d-1}x^{d-1}+\cdots+a_1x+a_0$ is given, in terms of its roots $\{r_1,r_2,\dots,r_d\}\subset\C$ (not necessarily distinct), by
\begin{equation}\label{eq:discrim}
	\Delta(A)=\prod_{i<j}(r_i-r_j)^2, \quad 1\leqslant i,j\leqslant d.
\end{equation}

It will be useful to remember that the $5$th cyclotomic polynomial is $\Phi_{5}(x)=x^{4} + x^{3} + x^{2} + x + 1$. We shall also define what a \emph{prime divisor} of a given polynomial is.

\textbf{Definition.} Let $A(x)\in\Z[x]$ be a polynomial. We say that a prime number $p$ is a \emph{prime divisor} of $A$ (or simply that $p$ \emph{divides} $A$) if there exists $m\in\Z$ such that $p$ divides $A(m)$.

\section{The main Theorem}\label{sec:mainTh}

We are now able to show that there exist infinitely many primes $\equiv 1\pmod{5}$. For this purpose, consider the polynomial 
\begin{equation*}
\Phi_{5}(x)=x^{4} + x^{3} + x^{2} + x + 1.
\end{equation*}
We will specifically show that every prime divisor $p$ of $\Phi_{5}(x)$ either belongs to the finite set 
\begin{equation*}
T=\{5\}
\end{equation*}
or satisfies $p\equiv 1 \pmod{5}$. To see this, consider the set $S:=\{1, 2, 3, 4\}$ and the values $\zeta^{s}$, with $s\in S$ and $\zeta:=e^{2\pi i/{5}}$, a $5$th primitive root of unity (thus a root of $\Phi_{5}(x)$). A simple calculation shows that $\Phi_{5}(x)$ can be written as
\begin{equation*}
\Phi_{5}(x)=\prod_{s\in S}\big(x-\zeta^{s}\big)=x^{4} + x^{3} + x^{2} + x + 1.
\end{equation*}
The discriminant of $\Phi_{5}(x)$ can be calculated\footnote{One way of calculating $\Delta(\Phi_{5})$ is via the resultant of $\Phi_{5}$ and $\Phi_{5}'$.} to be $\Delta(\Phi_{5})=5^{3}$.

Now, suppose that $p$ is a prime divisor of $\Phi_{5}$ such that $p$ does not belong to $T$. Next, consider a field $\F$ containing both the finite field $\F_p$ and $\zeta$\footnote{For instance, consider $\F=\F_{p^n}$ with a suitable integer $n\geqslant 1$ such that $\Phi_{5}$ has a root $\zeta$.}. Since $p$ divides $\Phi_{5}$, working in $\F$, there exists $a\in\Z$ such that 
\begin{equation*}
\Phi_{5}(a)=\prod_{s'\in S}\big(a-\zeta^{s'}\big)=0.
\end{equation*}
Since $\F$ is a field, there exists some $s\in S$ such that $a=\zeta^{s}$.

\textbf{Lemma.} \emph{The equality $\zeta^s=\zeta^{ps}$ holds in $\F$.}

\textit{Proof.} Observe that the following calculation holds in $\F$:
\begin{equation*}\label{eq:reproots}
\zeta^{s}=a=a^p=\zeta^{ps},
\end{equation*}
where we have used Fermat little theorem in the second equality. \ $\blacksquare$ 

Therefore, the fact that $\zeta^{s}=\zeta^{ps}$ means that $\zeta^{ps}=\zeta^{s}$ is a root of $\overline{\Phi_{5}(x)}\in\F[x]$. 

\textbf{Lemma.} \emph{$\zeta^{ps}$ is also a root of $\Phi_{5}(x)$ in $\Q(\zeta)$ $($the smallest subfield of $\C$ containing $\zeta)$.}

\textit{Proof.} Begin by noting that the value $\zeta^{ps}$ only depends on the value of $ps \bmod{5}$ since it only appears as an exponent of $\zeta$. Since $p$ does not divide $5$ and $s$ is coprime to $5$, $ps$ is coprime to $5$ (so $ps \bmod{5}$ is coprime to $5$) and hence $\zeta^{ps}$ is a primitive $5$th root of unity. Thus, $\zeta^{ps}$ is a root of $\Phi_{5}(x)$ in $\Q(\zeta)$. \ $\blacksquare$

\textbf{Lemma.} \emph{$\zeta^{ps}$ and $\zeta^{s}$ are the same root of $\Phi_{5}(x)$ in $\Q(\zeta)$.}

\textit{Proof.} If $\zeta^{ps}$ and $\zeta^{s}$ were two distinct roots of $\Phi_{5}(x)$ in $\Q(\zeta)$, we know because of the equality $\zeta^{s}=\zeta^{ps}$ that they would be the same in $\F$. Therefore, observing the expression of the discriminant, it follows that $\Delta(\Phi_{5} \bmod{p})=\Delta(\Phi_{5}) \bmod{p}=0$, so $p$ divides $\Delta(\Phi_{5})=5^{3}$. This is a contradiction with our choice of $p$. Thus, $\zeta^{ps}$ and $\zeta^{s}$ are in fact the same root of $\Phi_{5}(x)$ in $\Q(\zeta)$. \ $\blacksquare$

Therefore, the equality
\begin{equation}\label{eq:equality_in_zeta}
\zeta^{ps}=\zeta^{s}
\end{equation}
holds in $\Q(\zeta)$. 

\textbf{Lemma.} \emph{The fact that $\zeta^{ps}=\zeta^{s}$ holds in $\Q(\zeta)$ implies that $p \bmod{5}=1$.}

\textit{Proof.} Write the above equation in terms of $\theta:=\zeta^{s}$. This change yields
\begin{equation*}
\theta^{p}=\theta,
\end{equation*}
where observe that $\theta$ is also a primitive $5$th root of unity. Now, the right-hand side of the equation above does not depend on $p$. The left-hand side only depends on the value of $p\bmod{5}$, since $p$ only appears as an exponent of $\theta$. In conclusion, $\zeta^{ps}=\zeta^{s}$ only holds if $p \bmod{5}=1$, that is, if $p\equiv 1\pmod{5}$. \ $\blacksquare$

In conclusion, every prime divisor $p$ of $\Phi_{5}$ either belongs to the finite set 
\begin{equation*}
T=\{5\}
\end{equation*}
or satisfies $p\equiv 1 \pmod{5}$. In Section \ref{sec:properties} we will establish that the polynomial $\Phi_{5}$ has infinitely many prime divisors. But, from the remark above, all these prime divisors must be $\equiv 1\pmod{5}$ (except for those $p\in T$). This concludes the proof that there are infinitely many primes $\equiv 1 \pmod{5}$.

\section{Property of the polynomial \texorpdfstring{$\Phi_{5}(x)$}{Pkx}}\label{sec:properties}

We just need the following lemma to complete the proof of the main Theorem in Section \ref{sec:mainTh}.

\textbf{Lemma.} \emph{The cyclotomic polynomial $\Phi_{5}(x)\in\Z[x]$ has infinitely many prime divisors.}

\textit{Proof.} There is obviously at least one prime divisor of $\Phi_{5}$, since the case $\Phi_{5}(x)=x^{4} + x^{3} + x^{2} + x + 1=\pm 1$ only happens for a finite number of integer values of $x$. Now suppose $\Phi_{5}$ has only a finite number of prime divisors, say $p_1, p_2,\dots,p_k$ and let $Q:=p_1p_2\cdots p_k$. 

Observe that $\degpol(\Phi_{5})=4$ and $\Phi_{5}(0)=1\neq 0$. Then, $\Phi_{5}(Qx)=g(x)$ for some $g(x)\in\Z[x]$ of the form $1+c_1x+\cdots+c_{4}x^{4}$, $c_i\in\Z$, for every $1\leqslant i \leqslant 4$. Note that $Q$ divides every $c_i$. This polynomial $g$ must also have at least one prime divisor, say $p$, for the same reason as before. Therefore, $p$ divides $g(m)$ for some $m\in\Z$, and this implies that $p$ divides $\Phi_{5}(Qm)$. Since $m':=Qm\in\Z$, it follows that $p$ is a prime divisor of $\Phi_{5}$. But $p$ does not divide $Q$, since $p$ dividing $Q$ would mean that $p$ divides $c_i$, for every $1\leqslant i \leqslant 4$ (recall that $Q$ divides every $c_i$). This, together with the fact that $p$ divides $g(m)$, would imply that $p$ divides $g(m) - \sum_{i=1}^{4}c_im^i=1$, which means $p=1$, a contradiction. 

Now, $p$ is a prime divisor of $\Phi_{5}$, but $p$ is not any of the primes $p_1,p_2,\dots,p_k$, since we just proved that $p$ does not divide $p_1p_2\cdots p_k=Q$. Thus, we found a new prime divisor of $\Phi_{5}$ not in our list, so one concludes that $\Phi_{5}$ has infinitely many prime divisors. \ $\blacksquare$

\end{document}
