\documentclass[a4paper, 12pt]{article}
\usepackage{geometry}
\geometry{top=2.54cm, bottom=2.54cm, left=2.54cm, right=2.54cm, heightrounded}
\usepackage[utf8]{inputenc}
\usepackage[T1]{fontenc}
\usepackage[english]{babel} 
\usepackage{amsthm, amsmath, amssymb, amsfonts, amscd}
\usepackage{mathtools}
\usepackage{fancyhdr}
\usepackage{setspace}
\usepackage{hyperref}
\usepackage{graphicx}
\usepackage{ifluatex}
\usepackage{array}
\usepackage{multirow}
\usepackage[table]{xcolor}
\usepackage{tcolorbox}
\usepackage{mdframed}

% ---- Spacing ---- %
\linespread{1.0}
\onehalfspacing

% ---- Page header ---- %
\pagestyle{fancy}
\headheight 30pt
\fancyhf{}
\rhead{Page \thepage}
\lhead{\nouppercase{Infinite number of primes in $n+1$, $n\geqslant 0$}}
\headsep 1.5em

% ---- Link & URL style ---- %
\hypersetup{colorlinks, citecolor=green, linkcolor=blue, urlcolor=blue}
\urlstyle{rm}

% ---- Paragraph indentation style ---- %
\setlength{\parindent}{20pt}

% ---- Box ---- %
\definecolor{framecolor}{rgb}{0.5, 0.5, 0.5}
\definecolor{backcolor}{rgb}{0.94, 0.97, 1.0}
\newtcolorbox{mybox}{colback=backcolor, colframe=framecolor, boxrule=0.5pt, arc=3pt}

% ---- Theorems ---- %
\newtheorem{theorem}{Theorem}%[section]
\newtheorem{lemma}[theorem]{Lemma}

\theoremstyle{definition}
\newtheorem{definition}[theorem]{Definition}

% ---- Definitions ---- %
\renewcommand{\qedsymbol}{\rule{0.7em}{0.7em}} % Black box to end proofs. 
\newcommand{\Z}{\ensuremath{\mathbb{Z}}}
\newcommand{\Q}{\ensuremath{\mathbb{Q}}}
\newcommand{\C}{\ensuremath{\mathbb{C}}}
\newcommand{\F}{\ensuremath{\mathbb{F}}}
\newcommand{\Mod}[1]{\ (\mathrm{mod}\ #1)} % (mod k) sequence.
\DeclareUnicodeCharacter{03B6}{\zeta} % Define Unicode Character 'ζ'. 
\newcommand{\splitatcommas}[1]{%
	\begingroup
	\begingroup\lccode`~=`, \lowercase{\endgroup
		\edef~{\mathchar\the\mathcode`, \penalty0 \noexpand\hspace{0pt plus 1em}}%
	}\mathcode`,="8000 #1%
	\endgroup
}% Break long inline expressions after comma sign
\newsavebox{\equationbox} % Save box for storing the equation
\newcommand{\autosizeequation}[1]{%
	\sbox{\equationbox}{$\displaystyle #1$}% Store the equation in the box
	\ifdim \wd\equationbox > \textwidth % Check if the width exceeds text width
	\resizebox{\textwidth}{!}{$\displaystyle #1$}% Resize if too wide
	\else
	#1 % Otherwise, leave it as is
	\fi
}% Define a custom macro that checks the equation's length and resizes if necessary
\renewcommand\title{\textbf{The arithmetic progression $n+1$, $n\geqslant0$, contains infinitely many primes.\\ A Euclidean proof}}

\begin{document}
	\thispagestyle{plain}
	\pagenumbering{arabic}
	\sloppy
	%\begin{center}
	%	{\huge \bfseries \title\par}
	%	\vspace{1cm}
	%\end{center}
	\section{The arithmetic progression $n+1$, $n\geqslant0$, contains infinitely many primes. A Euclidean proof}

	The fact that the arithmetic progression $\equiv 1 \Mod{1}$, that is, $n+1$, $n\geqslant0$, contains infinitely many primes, stems from the fact that there exist infinitely many primes. For completeness we shall reproduce here Euclid's proof that there exist infinitely many primes.
	
\textbf{Theorem (Euclid).} \emph{There are infinitely many prime numbers.}

	\textit{Proof.} Suppose there are finitely many primes, say $p_1,p_2,\dots,p_m$. Our goal is to show that there exists yet another prime not in our list. Thus, consider $Q:=p_1p_2\cdots p_m+1$. This number has at least one prime divisor, $p$, since $Q>1$. If $p$ was one of the $p_i$ for $1\leqslant i \leqslant m$, then $p$ would divide $p_1p_2\cdots p_m$. Since $p$ also divides $Q$, we deduce that $p$ divides $Q-p_1p_2\cdots p_m=1$, so $p$ must be equal to $1$, a contradiction ($1$ is not a prime). Therefore, $p$ is a prime and it cannot be in our list, which is a contradiction again. \ $\blacksquare$
	
\end{document}
