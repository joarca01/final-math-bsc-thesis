\documentclass[a4paper, 12pt]{article}
\usepackage{geometry}
\geometry{top=2.54cm, bottom=2.54cm, left=2.54cm, right=2.54cm, heightrounded}
\usepackage[utf8]{inputenc}
\usepackage[T1]{fontenc}
\usepackage[english]{babel} 
\usepackage{amsthm, amsmath, amssymb, amsfonts, amscd}
\usepackage{mathtools}
\usepackage{fancyhdr}
\usepackage{setspace}
\usepackage{hyperref}
\usepackage{graphicx}
\usepackage{ifluatex}
\usepackage{array}
\usepackage{multirow}
\usepackage[table]{xcolor}
\usepackage{tcolorbox}
\usepackage{mdframed}

% ---- Spacing ---- %
\linespread{1.0}
\onehalfspacing
{progression_2k_alt} For this purpose, consider the polynomial 
\begin{equation*}
f(x):={poly}.
\end{equation*}

{section_alt}

{f0_factors_alt} Therefore, $f(c)$ is only divisible by primes $\equiv 1\pmod{{k}}$ (and by $p_{{index_prime}}={first_prime}$). \ $\blacksquare$

Finally, from the fact that $f(c)$ has every prime divisor $\equiv 1 \pmod{{k}}$ except for $p_{{index_prime}}={first_prime}$ it follows, mod ${k}$, that $f(c)=1\cdot1\cdots 1\cdot{ell}={ell}$ (note that ${ell}$ only appears once because $p_{{index_prime}}={first_prime}\equiv{ell}\pmod{{k}}$ and the fact that ${first_prime}^2$ does not divide $f(c)$). However, observe that $f(c)\equiv f(0)\equiv 1\pmod{{k}}$. This is a contradiction. Therefore, the arithmetic progression $\equiv {ell}\pmod{{k}}$ contains infinitely many primes.

{prime_divisors_alt} 

\end{document}