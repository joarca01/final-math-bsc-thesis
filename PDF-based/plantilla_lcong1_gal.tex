\documentclass[a4paper, 12pt]{article}
\usepackage[a4paper, total={165mm, 250mm}]{geometry}
\geometry{top=2.54cm, bottom=2.54cm, left=2.54cm, right=2.54cm}
\usepackage[utf8]{inputenc}
\usepackage{titlesec}
\usepackage[T1]{fontenc}
\usepackage[english]{babel} 
\usepackage{amsthm,amsmath,amssymb,amsfonts,amscd}
\usepackage{mathtools}
\usepackage{graphicx}
\usepackage{enumerate}
\usepackage[all]{xy}
\usepackage{booktabs}
\usepackage[usenames]{xcolor}
\usepackage{fancyhdr}
\usepackage{caption}
\usepackage{setspace}
\usepackage{cite}  
\usepackage{hyperref}
\usepackage{tikz-cd}
\usepackage{authblk}
\usepackage[normalem]{ulem}

\linespread{1.0}
\onehalfspacing

\pagestyle{fancy}
\headheight 30pt
\fancyhf{}
\rhead{Page \thepage}
\lhead{\nouppercase{Infinite number of primes in ${k}n+{ell}$, $n\geqslant 0$}}
\headsep 1.5em

\newtheorem*{theoremA}{Theorem A}
\newtheorem{theorem}{Theorem}%[section]

\newtheorem{proposition}[theorem]{Proposition}
\newtheorem{lemma}[theorem]{Lemma}
\newtheorem{corollary}[theorem]{Corollary}
\newtheorem{conjecture}[theorem]{Conjecture}

\theoremstyle{definition}
\newtheorem{definition}[theorem]{Definition}
\newtheorem{example}[theorem]{Example}

\theoremstyle{remark}
\newtheorem{remark}[theorem]{Remark}
\newtheorem*{remarknonumber}{Remark}
\newtheorem{observation}[theorem]{Observation}

\renewcommand{\qedsymbol}{\rule{0.7em}{0.7em}} % Black box to end proofs. 
\newcommand{\N}{\ensuremath{\mathbb{N}}}
\newcommand{\Z}{\ensuremath{\mathbb{Z}}}
\newcommand{\R}{\ensuremath{\mathbb{R}}}
\newcommand{\C}{\ensuremath{\mathbb{C}}}
\newcommand{\Q}{\ensuremath{\mathbb{Q}}}
\newcommand{\A}{\ensuremath{\mathbb{A}}}
\newcommand{\F}{\ensuremath{\mathbb{F}}}
\newcommand{\Mod}[1]{\ (\mathrm{mod}\ #1)} % (mod k) sequence.
\DeclareUnicodeCharacter{03B6}{\zeta} % Define Unicode Character 'ζ'. 

\definecolor{applegreen}{rgb}{0.55, 0.71, 0.0}

% ---- Figue, table and equation labelling ---- %
%\numberwithin{equation}{section}
%\numberwithin{figure}{section}
%\numberwithin{table}{section}
%\captionsetup[figure]{font=small, labelfont=large}

% ---- Link style ---- %
\hypersetup{colorlinks, citecolor=green, linkcolor=blue, urlcolor=blue}

% ---- Paragraph indentation style ---- %
\setlength{\parindent}{20pt}

\title{\textbf{The arithmetic progression ${k}n+{ell}$, $n\geqslant0$, contains infinitely many primes.\\ A proof using Galois Theory}}
\author[1]{Joan Arenillas i Cases}
%\affil[1]{Department of Mathematics, Universitat Autònoma de Barcelona}

\begin{document}
\maketitle
\sloppy
We will prove that the arithmetic progression $\equiv {ell} \Mod{{k}}$, that is, ${k}n+{ell}$, $n\geqslant0$, contains infinitely many primes. We will begin with some notation first.

Let $\zeta$ be a ${k}$-th primitive root of unity (for example $\zeta=e^{2\pi i/{{k}}}$) and let $K:=\Q(\zeta)$. We know that $\text{Gal}(K/\Q)\cong (\Z/{k}\Z)^\times$, the group of coprime residue classes modulo ${k}$. We shall also define what a \emph{prime divisor} of a given polynomial is. 
\begin{definition}
Let $f\in\Z[x]$ be a polynomial. We say that a prime number $p$ is a \emph{prime divisor} of $f$ if there exists $m\in\Z$ such that $p\mid f(m)$. We shall generally write $p\mid f$.
\end{definition}

We also need to give an expression for the discriminant of a polynomial. In terms of its roots $r_i$ (not necessarily distinct), the discriminant of a polynomial $A(x)=a_nx^n+\cdots+a_1x+a_0$ is given by
	\begin{equation}\label{discrim}
		D(A)=a_n^{2n-2}\prod_{i<j}(r_i-r_j)^2.
	\end{equation}

A preliminar result to show that there exist infinitely many primes $\equiv {ell}\Mod{{k}}$ is also needed:

\begin{lemma}\label{ThSchur}
The cyclotomic polynomial $\Phi_{{k}}(x)\in\Z[x]$ has infinitely many prime divisors.
\end{lemma}
\begin{proof}
Note that $\Phi_{{k}}(0)={indep_coef_cyclotomic}\neq 0$. There is obviously at least one prime divisor of $\Phi_{{k}}$, since the case $\Phi_{{k}}(x)={poly}=\pm 1$ only happens for a finite number of integer values of $x$. Now suppose $\Phi_{{k}}$ has only a finite number of prime divisors, say $p_1,\dots,p_k$. Let $Q:=p_1\cdots p_k$. Observe that $\text{deg}(\Phi_{{k}})={eulersphi_k}$. Then $\Phi_{{k}}({indep_coef_cyclotomic}\cdot Qx)={indep_coef_cyclotomic}\cdot g(x)$ for some $g(x)\in\Z[x]$ of the form $1+c_1x+\cdots+c_{{eulersphi_k}}x^{{eulersphi_k}}$, $c_i\in\Z$, satisfying $Q\mid c_i$ for every $1\leqslant i \leqslant {eulersphi_k}$. This polynomial $g$ must also have at least one prime divisor, say $p$, for the same reason as before. Therefore, $p\mid g(m)$ for some $m\in\Z$ and this implies $p\mid {indep_coef_cyclotomic}\cdot g(m)\Rightarrow p\mid \Phi_{{k}}({indep_coef_cyclotomic}\cdot Qm)$. Since $m':={indep_coef_cyclotomic}\cdot Qm\in\Z$, $p\mid f$. But $p\nmid Q$, since $p\mid Q\mid c_i, \,\forall i$ and $p\mid g(m)$ would imply $p\mid g(m) - \sum_{i=1}^{{eulersphi_k}}c_im^i=1$, which means $p=1$, a contradiction. Now $p$ is a prime divisor of $\Phi_{{k}}$ but $p$ is not any of the primes $p_1,\dots,p_k$, for otherwise $p\mid p_1\cdots p_k=Q$. Thus, we found a new prime divisor of $\Phi_{{k}}$ not in our list. Since this argument can be repeated indefinitely, one concludes that $\Phi_{{k}}$ has infinitely many prime divisors.
\end{proof}

We are now ready to begin the proof. In light of Schur's Theorem \cite{Murty}, we will begin by proving the following proposition.

\begin{proposition}[\textbf{Schur}]\label{th4murty}
Consider the subgroup $H=\{1\}$ of $G:=(\Z/{k}\Z)^\times$. Then there exists an irreducible polynomial $f\in\Z[x]$ such that all of the prime divisors of $f$, with a finite number of exceptions, belong to the residue class of $H$. In fact, $f=\Phi_{{k}}$, the ${k}$-th cyclotomic polynomial.
\end{proposition}
\begin{proof}
In the following, we will consider this tower of extensions over $\Q$:
	\[
	\begin{tikzcd}	
	K=\Q(\zeta) \arrow[d, dash] \\ 
	L:=\Q(\zeta)^H \arrow[d, dash] \\ 
	\Q,
	\end{tikzcd}
	\]
\vspace{0.2cm}
where we have defined $L$ to be the fixed field of $H$. Since $G$ is finite, we can list the coset representatives of $H$ in $G$. The elements of $G$ are ${coprimes_list}$, and the coset representatives can be chosen to be ${coset_reps_list}$, which we denote by $m_i$, $1\leqslant i \leqslant {num_coset_reps}$. Set $\eta:=h(\zeta)$ and $h(z):=z\in\Z[z]$, so $\eta=\zeta$. Now, define $\eta_i:=h(\zeta^{m_i})=\zeta^{m_i}$. The values $\zeta^{m_i}$ are ${f_polynomial_roots_list}$, which are clearly distinct. Therefore $\Q(\eta)=\Q(\zeta)=L$.

We now define the polynomial
\begin{equation}\label{polinomi}
f(x):=\prod_{i=1}^{{num_coset_reps}}(x-\eta_i)=\prod_{i=1}^{{num_coset_reps}}\big(x-h(\zeta^{m_i})\big)=\prod_{i=1}^{{num_coset_reps}}\big(x-\zeta^{m_i}\big),
\end{equation}
which is irreducible over $\Q[x]$ since $\eta_i$, $1\leqslant i \leqslant {num_coset_reps}$, are the distinct conjugates of $\eta$. We can explicitely write $f(x)$. A simple calculation yields
\begin{align*}
f(x)&=\prod_{i=1}^{{num_coset_reps}}\Big(x-e^\frac{2\pi im_i}{{k}}\Big)\\
&={poly}\\
&=\Phi_{{k}}(x).
\end{align*}
Note that $\Phi_{{k}}(x)$ is the minimal polynomial of $\zeta$, so $[\Q(\zeta):\Q]=\text{deg}(\Phi_{{k}})=\varphi({k})={eulersphi_k}$ ($\varphi$ denotes Euler's phi function). Therefore, we deduce that $[\Q(\eta):\Q]=1$, as expected (since $H=\{1\}$). 
 
Now, suppose that $p$ is a prime divisor of $f(x)=\Phi_{{k}}(x)$ such that $p\nmid {k}$ and $p\nmid D(f)={factor_discriminant}$, where $D(f)$ is the discriminant of $f$. Therefore, $p\neq{p_exclusions_list}$. The existence of such $p$ is guaranteed by Lemma \ref{ThSchur}, since we only excluded a finite number of values of $p$.

We shall now work in the field $\F_p(\zeta)\cong\F_p[x]/(\Phi_{{k}}(x))$. Since $p\mid f$ there exists $a\in\Z$ such that 
\begin{equation*}
f(a)=\prod_{i=1}^{{num_coset_reps}}\big(a-\zeta^{m_i}\big)=0.
\end{equation*}
Now, since $\F_p(\zeta)$ is a field, there exists $i$ such that $a=\zeta^{m_i}$ and the following calculation holds in this field:
\begin{equation}\label{arrelsrep}
\zeta^{m_i}=a=a^p=\zeta^{pm_i},
\end{equation}
where we have used Fermat's little theorem in the second equality. Therefore, equality \eqref{arrelsrep} means that $\zeta^{m_i}$ and $\zeta^{pm_i}$ are roots of $\overline{f(x)}\in\F_p(\zeta)[x]$. 

Recall that in the field $\Q(\zeta)$ the roots of $f(x)=\Phi_{{k}}(x)$ are ${f_polynomial_roots_list}$. If any two roots were equal in the field $\F_p(\zeta)$, then $D(f \Mod{p})=D(f) \Mod{p}=0$, so $p\mid D(f)$, observing expression \eqref{discrim}. Since we chose $p\nmid D(f)={factor_discriminant}$, then $D(f)\not\equiv 0 \Mod{p} \Rightarrow D(f \Mod{p})\neq 0$ and so $f$ has no repeated roots$\Mod{{p}}$. That is, in the field $\F_p(\zeta)$ the polynomial $f$ has also ${num_coset_reps}$ distinct roots.

In $\F_p(\zeta)$ we already know that $\zeta^{pm_i}$ is a root of $f(x)$, since $\zeta^{pm_i}=\zeta^{m_i}$, which is a root of $f(x)$ by construction. In $K$, $\zeta^{pm_i}$ is also a root of $f(x)$, \sout{since $p$ only appears as an exponent of $\zeta$, the quantity $\zeta^{pm_i}$ only depends on the value of $pm_i \Mod{{k}}$, that is, in the residue of $pm_i/{k}$. Now, $p,m_i\nmid {k}$ implies $(pm_i,{k})=1$. Let $r,d\in\Z$ be such that $pm_i={k}d+r$. Then $(r,{k})=1$: otherwise $(r,{k})\neq 1\Rightarrow (r,{k})=s>1$, for $s\in\Z$. Then $s\mid r$ and $s\mid {k}\Rightarrow s\mid {k}d \Rightarrow s\mid {k}d+r$. This means $s\mid pm_i\Rightarrow (pm_i,{k})=s>1$, a contradiction. Therefore $(r,{k})=1$ with $1\leqslant r \leqslant{k}$ and we have $\zeta^{pm_i}=\zeta^{r}$, so $\zeta^{pm_i}$ must be one of the $\zeta^{m_i}$ in the product \eqref{polinomi}, which runs over $\{{coprimes_list}\}$. Therefore, $\zeta^{pm_i}$ is a root of $f(x)$ in $\Q(\zeta)$.}

since $\zeta^{pm_i}=\zeta^{m_i}$ in the field $\Q(\zeta)$. If this was not true, then we would have two distinct roots of $f(x)$, which would be the same$\Mod{p}$ (because of \eqref{arrelsrep}), thus $f \Mod{p}$ would have repeated roots, which is impossible since we already deduced $D(f \Mod{p})\neq 0$. Therefore, the equality
\begin{equation*}
\zeta^{pm_i}=\zeta^{m_i}
\end{equation*}
holds in $\Q(\zeta)$. Now write the above equation in terms of $\rho:=\zeta^{m_i}$. This change yields
\begin{equation}
\rho^{p}=\rho \label{equality_in_zeta}.
\end{equation}
The right hand side of the equation above does not depend on $p$. The left hand side only depends on the value of $p\Mod{{k}}$, since $p$ only appears as an exponent of $\rho$. But expression \eqref{equality_in_zeta} only holds if $p \Mod{{k}}\in H=\{1\}$. Therefore, $p$ belongs to the residue class of $H$.
\end{proof}

We will need the following corollary to show that there exist infinitely many primes $\equiv {ell}\Mod{{k}}$.

\begin{corollary}\label{CyclotomicCorollary}
Consider $\Phi_{{k}}(x)={cyclotomic_polynomial}$, the ${k}$-th cyclotomic polynomial. Then all the prime divisors of $\Phi_{{k}}$ are either $\equiv 1\Mod{{k}}$ or divide ${k}$.
\end{corollary}
\begin{proof}
Start by recalling that the cyclotomic polynomials are irreducible. Now, Proposition \ref{th4murty} tells us that all prime divisors of $\Phi_{{k}}$, except finitely many, belong to the residue class of $H=\{1\}$. In fact, the exceptions are: either when $p\mid {k}$ or $p\mid D(\Phi_{{k}})$. Note that the only primes $p$ that divide $D(\Phi_{{k}})={factor_discriminant_cyclo}$ are those $p$ such that $p\mid {k}$, that is, the only exceptions are $p={primes_div_k_list}$. In conclusion, if $p$ is a prime divisor of $\Phi_{{k}}$, it always happens that $p\equiv 1\Mod{{k}}$ or $p={primes_div_k_list}$.
\end{proof}

Lastly, the following Corollary gives us the desired result.
\begin{corollary}
	There are infinitely many primes $\equiv {ell} \Mod{{k}}$.
\end{corollary}
\begin{proof}
Begin by noting that there are finitely many numbers dividing {k}, namely ${primes_div_k_list}$. Now, from Corollary \ref{CyclotomicCorollary}, it follows that all prime divisors of $\Phi_{{k}}$ (except from those primes dividing ${k}$, which are finite) must be $\equiv 1\Mod{{k}}$. In Lemma \ref{ThSchur} we proved that the polynomial $\Phi_{{k}}$ has inifinitely many prime divisors. This infinite number of primes must be, at the same time, except finitely many, $\equiv 1\Mod{{k}}$. This concludes the prove that the arithmetic progression ${k}n+{ell}$, $n\geqslant 0$, contains infinitely many primes.
\end{proof}

\newpage
	
\bibliographystyle{amsalpha}
\bibliography{references}

\end{document}