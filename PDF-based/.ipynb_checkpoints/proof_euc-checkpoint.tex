\documentclass[a4paper, 12pt]{article}
\usepackage[a4paper, total={165mm, 250mm}]{geometry}
\geometry{top=2.54cm, bottom=2.54cm, left=2.54cm, right=2.54cm}
\usepackage[utf8]{inputenc}
\usepackage{titlesec}
\usepackage[T1]{fontenc}
\usepackage[english]{babel} 
\usepackage{amsthm,amsmath,amssymb,amsfonts,amscd}
\usepackage{mathtools}
\usepackage{graphicx}
\usepackage{enumerate}
\usepackage[all]{xy}
\usepackage{booktabs}
\usepackage[usenames]{xcolor}
\usepackage{fancyhdr}
\usepackage{caption}
\usepackage{setspace}
\usepackage{cite}  
\usepackage{hyperref}
\usepackage{tikz-cd}
\usepackage{authblk}
\usepackage[normalem]{ulem}

\linespread{1.0}
\onehalfspacing

\pagestyle{fancy}
\headheight 30pt
\fancyhf{}
\rhead{Page \thepage}
\lhead{\nouppercase{Infinite number of primes in $12n+5$, $n\geqslant 0$}}
\headsep 1.5em

\newtheorem*{theoremA}{Theorem A}
\newtheorem{theorem}{Theorem}%[section]

\newtheorem{proposition}[theorem]{Proposition}
\newtheorem{lemma}[theorem]{Lemma}
\newtheorem{corollary}[theorem]{Corollary}
\newtheorem{conjecture}[theorem]{Conjecture}

\theoremstyle{definition}
\newtheorem{definition}[theorem]{Definition}
\newtheorem{example}[theorem]{Example}

\theoremstyle{remark}
\newtheorem{remark}[theorem]{Remark}
\newtheorem*{remarknonumber}{Remark}
\newtheorem{observation}[theorem]{Observation}

\renewcommand{\qedsymbol}{\rule{0.7em}{0.7em}} % Black box to end proofs. 
\newcommand{\N}{\ensuremath{\mathbb{N}}}
\newcommand{\Z}{\ensuremath{\mathbb{Z}}}
\newcommand{\R}{\ensuremath{\mathbb{R}}}
\newcommand{\C}{\ensuremath{\mathbb{C}}}
\newcommand{\Q}{\ensuremath{\mathbb{Q}}}
\newcommand{\A}{\ensuremath{\mathbb{A}}}
\newcommand{\F}{\ensuremath{\mathbb{F}}}
\newcommand{\Mod}[1]{\ (\mathrm{mod}\ #1)} % (mod k) sequence.
\DeclareUnicodeCharacter{03B6}{\zeta} % Define Unicode Character 'ζ'. 

\definecolor{applegreen}{rgb}{0.55, 0.71, 0.0}

% ---- Figue, table and equation labelling ---- %
%\numberwithin{equation}{section}
%\numberwithin{figure}{section}
%\numberwithin{table}{section}
%\captionsetup[figure]{font=small, labelfont=large}

% ---- Link style ---- %
\hypersetup{colorlinks, citecolor=green, linkcolor=blue, urlcolor=blue}

% ---- Paragraph indentation style ---- %
\setlength{\parindent}{20pt}

\title{\textbf{The arithmetic progression $12n+5$, $n\geqslant0$, contains infinitely many primes.\\ An elementary proof}}
\date{\vspace{-5ex}}
\author[1]{Joan Arenillas i Cases}
%\affil[1]{Department of Mathematics, Universitat Autònoma de Barcelona}

\begin{document}
\maketitle
\sloppy
We will prove that the arithmetic progression $\equiv 5 \Mod{12}$ contains infinitely many primes. Equivalently, we will see that there are infinitely many primes of the form $12n+5$, $n\geqslant0$. To follow the proof, one must recall the expression for the discriminant of a polynomial. In terms of its roots $r_1,r_2,\dots,r_m$ (not necessarily distinct), the discriminant of a polynomial $A(x)=a_nx^n+\cdots+a_1x+a_0$ is given by
    \begin{equation}\label{discrim}
		\Delta(A)=a_n^{2n-2}\prod_{i<j}(r_i-r_j)^2.
	\end{equation}

We shall also define what a \emph{prime divisor} of a given polynomial is. 
\begin{definition}
Let $f\in\Z[x]$ be a polynomial. We say that a prime number $p$ is a \emph{prime divisor} of $f$ if there exists $m\in\Z$ such that $p$ divides $f(m)$. We shall generally write $p\mid f$.
\end{definition}

It will also be handy to have in mind that the $12$-th cyclotomic polynomial is $\Phi_{12}(x)=x^{4} - x^{2} + 1$. We are now in position of starting the proof.

\begin{proposition}\label{th4murty}
Consider the polynomial $f(x)=x^{2} + 286 x + 20593$. Suppose $p$ is a prime divisor of $f$. Then $p \Mod{12}$ belongs to $H:=\{1,5\}$, except for finitely many exceptions.  
\end{proposition} 
\begin{proof}
Consider the set $S=\{1, 7\}$ and the values $h(\zeta^{m_i}):=-(12-\zeta^{m_i})(12-\zeta^{5m_i})$, with $m_i\in S$, $1 \leqslant i \leqslant 2$, and $\zeta:=e^{2\pi i/{12}}$, a $12$-th primitive root of unity (thus a root of $\Phi_{12}(x)$). A simple calcualtion shows that $f(x)$ can be written as

\begin{equation*}
\binoppenalty=10000 % no line break after \cdot
\mathchardef\plus=\mathcode`+ % save the mathcode
\mathcode`+="8000 % make the plus math active
\thinmuskip=0mu
\begingroup\lccode`~=`+
  \lowercase{\endgroup\def~}{%
    \plus\allowbreak\hspace{3pt plus 2pt}%
}
\hphantom{f(x)=\prod_{i=1}^{2}\big(}
\parbox{0.8\displaywidth}{\linespread{1.1}\selectfont
  \makebox[0pt][r]{$f(x)=\displaystyle\prod_{i=1}^{2}\big($}$x-h(\zeta^{m_i})\big)=x^{2} + 286 x + 20593.$
}
\end{equation*}

%\begin{align}\label{polinomi}
%f(x)&=\prod_{i=1}^{2}\big(x-h(\zeta^{m_i})\big)\\
%&=x^{2} + 286 x + 20593.
%\end{align}

The roots of $f(x)$ are $h(\zeta^{m_i})$ for $m_i \in S$. Their values are $12 ζ^{3} - 143, -12 ζ^{3} - 143$ so that the discriminant of $f(x)$ can be easily calculated to be $\Delta(f)=-1 \cdot 2^{6} \cdot 3^{2}$, following expression \eqref{discrim}.

Now, suppose that $p$ is a prime divisor of $f$ such that $p$ does not divide $12$ and $\Delta(f)=-1 \cdot 2^{6} \cdot 3^{2}$. Therefore, $p\neq2, 3$.

We shall now work in the field $\F_p(\zeta)$, which is the smallest field that contains both the finite field $\F_p$ and $\zeta$. Since $p$ divides $f$ there exists $a\in\Z$ such that 
\begin{equation*}
f(a)=\prod_{i=1}^{2}\big(a-h(\zeta^{m_i})\big)=0.
\end{equation*}
Now, since $\F_p(\zeta)$ is a field, there exists $i$ such that $a=h(\zeta^{m_i})$ and the following calculation holds in this field:
\begin{align}\label{arrelsrep}
h(\zeta^{m_i})&=a=a^p=h(\zeta^{m_i})^p=-(12-\zeta^{m_i})^p(12-\zeta^{5m_i})^p\notag \\ 
&=-(12^p-\zeta^{pm_i})(12^p-\zeta^{5pm_i})=-(12-\zeta^{pm_i})(12-\zeta^{5pm_i})=h(\zeta^{pm_i}),
\end{align}
where we have used Fermat's little theorem in the second equality. The fifth equality, on the other hand, relies on the fact that $\text{char}(\F_p(\zeta))=p$ (so that $(c+d)^p=c^p+d^p$ for every $c,d \in \F_p(\zeta)$) and the following one, on Fermat's little theorem. Therefore, equality \eqref{arrelsrep} means that $h(\zeta^{pm_i})=h(\zeta^{m_i})$ is also a root of $\overline{f(x)}\in\F_p(\zeta)[x]$. 

We will now see that $h(\zeta^{pm_i})$ is also a root of $f(x)$. Begin by noting that the value $h(\zeta^{pm_i})$ only depends on the value of $pm_i \Mod{12}$ since it only appears as an exponent of $\zeta$. Now, the fact that neither $p$ nor $m_i$ divide $12$ implies that $(pm_i,12)=1$. Let $r,d\in\Z$ be such that $pm_i=12d+r$. Then $(r,12)=1$: otherwise $(r,12)\neq 1$ implies $(r,12)=s>1$, for $s\in\Z$. Then, since $s$ divides $r$ and $s$ divides $12$, $s$ divides $12d$ so $s$ divides $12d+r$. This means that $s$ divides $pm_i$, which implies $(pm_i,12)=s>1$, a contradiction. Therefore, $(r,12)=1$ with $1\leqslant r < 12$ and we have that $h(\zeta^{pm_i})=h(\zeta^{r})$, with $\zeta^r$ being a primitive $12$-th root of unity.

There are now only two options: either $r\in S$ (therefore $h(\zeta^{pm_i})$ is obviously a root of $f(x)$) or $r\equiv 5m_i\Mod{12}$. In the latter case, let us prove that $h(\zeta^{5m_i})=h(\zeta^{m_i})$ for $m_i\in S$, so $h(\zeta^{r})=h(\zeta^{5m_i})=h(\zeta^{m_i})$ is also a root of $f(x)$. Indeed,
\begin{align*}
h(\zeta^{5m_i})&=-(12-\zeta^{5m_i})(12-\zeta^{5^2m_i})=-(12-\zeta^{5^2m_i})(12-\zeta^{5m_i})\\
&=-(12-\zeta^{m_i})(12-\zeta^{5m_i})=h(\zeta^{m_i}),
\end{align*}
where we have used that $\zeta^{5^2m_i}$ only depens on the value of $5^2m_i \Mod{12}$ and the fact that $5^2\equiv 1\Mod{12}$.

Therefore, $h(\zeta^{pm_i})=h(\zeta^{r})$ is always a root of $f(x)$ in $\Q(\zeta)$ (the smallest field that contains $\Q$ and $\zeta$). If $h(\zeta^{pm_i})$ and $h(\zeta^{m_i})$ were two distinct roots of $f(x)$ in $\Q(\zeta)$, we know because of \eqref{arrelsrep} that they would be the same in $\F_p(\zeta)$. Therefore, $\Delta(f \Mod{p})=\Delta(f) \Mod{p}=0$, so $p$ divides $\Delta(f)$, observing expression \eqref{discrim}. This is a contradiction with our choice of $p$. Thus, $h(\zeta^{pm_i})$ and $h(\zeta^{m_i})$ are in fact the same root of $f(x)$ in $\Q(\zeta)$.

Therefore, the equality
\begin{equation*}
\big({12}-\zeta^{pm_i}\big)\big({12}-\zeta^{5pm_i}\big)=\big({12}-\zeta^{m_i}\big)\big({12}-\zeta^{5m_i}\big)
\end{equation*}
holds in $\Q(\zeta)$. Now write the above equation in terms of $\theta:=\zeta^{m_i}$. This change yields
\begin{align}
-\big(\theta^{p}+\theta^{5p}\big){12}+\theta^{(1+5)p}&=-\big(\theta+\theta^{5}\big){12}+\theta^{1+5}, \notag \\
-12\big(\theta^{p}+\theta^{5p}\big)+\theta^{6p}&=-12\big(\theta+\theta^{5}\big)+\theta^{6}\label{equality_in_zeta}.
\end{align}
The right hand side of the equation above does not depend on $p$. The left hand side only depends on the value of $p\Mod{12}$, since $p$ only appears as an exponent of $\theta$. The above equality gives information about $p$, which is what we are interested in. We want to see that the fact that \eqref{equality_in_zeta} holds implies that $p \Mod{12}\in H$. To do this, we will check every other value of $p \Mod{12}$ (that is, $p \Mod{12}\notin H$) and conclude that \eqref{equality_in_zeta} is not true in $\Q(\theta)$ in those cases. Therefore, we shall see the following: if $p \Mod{12}\in G\setminus H=\{7, 11\}$, then $h(\theta^p)\neq h(\theta)$. This will automatically imply what we want prove: since \eqref{equality_in_zeta} holds, $p \Mod{12}\in H$. To see this, rewrite \eqref{equality_in_zeta} as

\begin{equation}\label{equalityQ}
-12\big(\theta^{p}+\theta^{5p}\big)+\theta^{6p}+12\big(\theta+\theta^{5}\big)-\theta^{6}=0
\end{equation}

and trade $\theta$ for $x$, since the condition \eqref{equalityQ} in $\Q(\theta)$ is equivalent to the condition

\begin{equation}\label{polycheck0}
-12\big(x^p+x^{5p}\big)+x^{6p}+12\big(x+x^{5}\big)-x^{6}=0
\end{equation}

in $\Q[x]/(\Phi_{12}(x))\cong\Q(\theta)$ ($\Phi_{12}(x)$ is also the minimal polynomial of $\theta$ since $\theta=\zeta^{m_i}$ is a primitive $12$-th root of unity). We will explicitely write the case $p=7$ (the remaining values of $p \Mod{12}\in G\setminus H$ are left as an exercise to the reader). With this value of $p$, equation \eqref{polycheck0} becomes

\begin{align*}
A(x)&:=-12\big(x^{7}+x^{35}\big)+x^{42}+12\big(x+x^{5}\big)-x^{6}\\
&=x^{42} - 12 x^{35} - 12 x^{7} - x^{6} + 12 x^{5} + 12 x=0.
\end{align*}
If we recall that $\Q(\theta)\cong\Q[x]/(\Phi_{12}(x))$, the above equation is equivalent to $A(x)$ being a multiple of the $12$-th cyclotomic polynomial, $\Phi_{12}(x) = x^{4} - x^{2} + 1$. Therefore, we are interested in showing that the residue $R(x)$ of the division $A(x)/\Phi_{12}(x)$ satisfies $R(x)\neq 0$, from which our result will follow. A simple Euclidean divison of polynomials shows that $A(x)=(x^{38} + x^{36} - x^{32} - 12 x^{31} - x^{30} - 12 x^{29} + x^{26} + 12 x^{25} + x^{24} + 12 x^{23} - x^{20} - 12 x^{19} - x^{18} - 12 x^{17} + x^{14} + 12 x^{13} + x^{12} + 12 x^{11} - x^{8} - 12 x^{7} - x^{6} - 12 x^{5} - 12 x^{3} + 12 x)\cdot\Phi_{12}(x)+(24 x^{3})$, so $R(x)=24 x^{3} \neq 0$. Therefore, equality \eqref{equality_in_zeta} implies that $p \Mod{12} \in H$. Thus, $p\equiv 1,5\Mod{12}$.
\end{proof}

We can now prove that

\begin{proposition}
There exist infinitely many primes $\equiv 5\Mod{12}$.
\end{proposition}
\begin{proof}
Consider the polynomial $f(x)=x^{2} + 286 x + 20593$. To prove that there exist infinitely many primes $\equiv 5 \Mod{12}$ we can procede by contradiction. Suppose there are finitely many primes $\equiv 5\Mod{12}$ and denote them by $p_1,\dots,p_m$. Since $5, 17 \equiv 5 \Mod{12}$, we can write the list as $5, 17, p_{3},\dots, p_m$ (so $p_{2}:=17$). Now let $Q:={2 \cdot 3 \cdot 5}\cdot p_{3}\cdots p_m$. Consider the following congruence equation system, where we try to solve for $c\in\Z$:
	\begin{align*}
		c&\equiv 7\Mod{17^2},\\
		c&\equiv 0\Mod{12Q}.
	\end{align*}
The Chinese Remainder Theorem guarantees the existence of a solution to this system (unique$\Mod{17^2\cdot12Q}$) since $17^2$ does not divide $12Q$. It follows that $f(c)\equiv f(7)=22644\equiv 17\cdot6\Mod{17^2}$ and $f(c)\equiv f(0)=20593\Mod{12Q}$. 

Now, since $f(0)=20593$ is only divisible by primes $\equiv 1\Mod{12}$, we deduce that $f(c)$ is only divisible by primes $\equiv 1\Mod{12}$ and by $p_{2}=17\equiv 5\Mod{12}$. Let's see why. Proposition \ref{th4murty} guarantees that every prime divisor $r$ of $f(c)$ is $r=2, 3$ or satisfies $r\equiv 1,5\Mod{12}$. If $r\equiv 1\Mod{12}$ or $r=p_{2}=17$ our claim is obvious. Thus, suppose $r=2, 3$ or $r=5, \text{\texttt{,}} p_{3},\dots,p_m$. This means that $r$ divides $12Q$. Now, since $c\equiv 0\Mod{12Q}$ we deduce that $c\equiv 0\Mod{r}$, so that $f(c)\equiv f(0)=20593\Mod{r}$. That would mean that $f(0)\equiv f(c)\equiv 0\Mod{r}$, since $r$ is a divisor of $f(c)$. But this is a contradiction, since $f(0)=20593$ is only divisible by primes $\equiv 1\Mod{12}$ and $r$ is not one of them: we know that either $r$ divides $12$ (which implies $r\equiv 0\not\equiv 1\Mod{12}$) or $r$ divides $\Delta(f)=-1 \cdot 2^{6} \cdot 3^{2}$ (which implies $r\not\equiv 1\Mod{12}$) or $r=5, \text{\texttt{,}} p_{3},\dots,p_m$ (but all the $p_i$'s are $\equiv 5 \not\equiv 1\Mod{12}$). Therefore, $f(c)$ is only divisible by primes $\equiv 1\Mod{12}$ and by $p_{2}=17$.

Since $17^2$ does not divide $f(7)=17\cdot1332$, one deduces that $17^2$ does not divide $f(c)$ from the fact that $f(c)\equiv f(7)\Mod{17^2}$. In fact, $17$ only divides $f(c)$ once, since $f(c)\equiv f(7)\Mod{17}$ and $f(7)=17\cdot1332$. Now, recall that $f(c)$ has every prime divisor $\equiv 1 \Mod{12}$ except for $p_{2}$, which is $\equiv 5\Mod{12}$. It follows,$\Mod{12}$, that $f(c)=1\cdot1\cdots 1\cdot5=5$ (observe that 5 only appears once). But we know that $f(c)\equiv f(0)=20593\equiv 1\Mod{12}$, since $f(0)$ is only divisible by primes $\equiv 1\Mod{12}$. That is a contradiction. Therefore, the arithmetic progression $\equiv 5\Mod{12}$ contains infinitely many primes.
\end{proof}

\newpage

%\thispagestyle{empty}
%\bibliographystyle{amsalpha}
%\bibliography{references}

\end{document}