
% ---- Page header ---- %
\pagestyle{fancy}
\headheight 30pt
\fancyhf{}
\rhead{Page \thepage}
\lhead{\nouppercase{Infinite number of primes in ${original_k}n+{original_ell}$, $n\geqslant 0$}}
\headsep 1.5em

% ---- Link & URL style ---- %
\hypersetup{colorlinks, citecolor=green, linkcolor=blue, urlcolor=blue}
\urlstyle{rm}

% ---- Paragraph indentation style ---- %
\setlength{\parindent}{20pt}

% ---- Box ---- %
\definecolor{framecolor}{rgb}{0.5, 0.5, 0.5}
\definecolor{backcolor}{rgb}{0.94, 0.97, 1.0}
\newtcolorbox{mybox}{colback=backcolor, colframe=framecolor, boxrule=0.5pt, arc=3pt}

% ---- Theorems ---- %
\newtheorem{theorem}{Theorem}%[section]
\newtheorem{lemma}[theorem]{Lemma}

\theoremstyle{definition}
\newtheorem{definition}[theorem]{Definition}

% ---- Definitions ---- %
\renewcommand{\qedsymbol}{\rule{0.7em}{0.7em}} % Black box to end proofs. 
\newcommand{\Z}{\ensuremath{\mathbb{Z}}}
\newcommand{\Q}{\ensuremath{\mathbb{Q}}}
\newcommand{\C}{\ensuremath{\mathbb{C}}}
\newcommand{\F}{\ensuremath{\mathbb{F}}}
\DeclareMathOperator{\degpol}{deg} % Degree of a polynomial.
\DeclareUnicodeCharacter{03B6}{\zeta} % Define Unicode Character 'ζ'. 
\newcommand{\splitatcommas}[1]{%
  \begingroup
  \begingroup\lccode`~=`, \lowercase{\endgroup
    \edef~{\mathchar\the\mathcode`, \penalty0 \noexpand\hspace{0pt plus 1em}}%
  }\mathcode`,="8000 #1%
  \endgroup
}% Break long inline expressions after comma sign
\newsavebox{\equationbox} % Save box for storing the equation
\newcommand{\autosizeequation}[1]{%
	\sbox{\equationbox}{$\displaystyle #1$}% Store the equation in the box
	\ifdim \wd\equationbox > \textwidth % Check if the width exceeds text width
	\resizebox{\textwidth}{!}{$\displaystyle #1$}% Resize if too wide
	\else
	#1 % Otherwise, leave it as is
	\fi
}% Define a custom macro that checks the equation's length and resizes if necessary
\renewcommand\title{\textbf{The arithmetic progression ${original_k}n+{original_ell}$, $n\geqslant0$, contains infinitely many primes.\\ A Euclidean proof}}

\begin{document}
\thispagestyle{plain}
\pagenumbering{arabic}
\sloppy
\begin{center}
{\huge \bfseries \title\par}
\vspace{1cm}
\end{center}

\begin{mybox}
{\Large \textbf{About this document}}
\vspace{0.3cm}\\
\textit{This file has been automatically generated for the user-supplied arithmetic progression. The code behind this document can be found in the url \url{https://github.com/joarca01/final-math-bsc-thesis}, and has been developed as part of a BSc Thesis in Mathematics by Joan Arenillas i Cases at the Autonomous University of Barcelona. The above link also provides full access to the complete Thesis. Please use \href{mailto:joanarenillas01@gmail.com}{\nolinkurl{joanarenillas01@gmail.com}} to report any typo or express any suggestions.}
\end{mybox}
\vspace{0.5cm}

By the Chinese Remainder Theorem, a prime $p$ will be $\equiv{original_ell}\pmod{{original_k}}$ if and only if $p\equiv{original_ell}\pmod{2}$ and $p\equiv{original_ell}\pmod{{k}}$. Since ${original_ell}$ is odd, the first condition is equivalent to $p\equiv 1\pmod{2}$, which is trivially satisfied by every prime (except for $2$). Therefore, proving that there exist infinitely many primes $\equiv{original_ell}\pmod{{original_k}}$ is equivalent to showing there exist infinitely many primes $\equiv {ell}\pmod{{k}}$, so we will instead proof this latter statement.