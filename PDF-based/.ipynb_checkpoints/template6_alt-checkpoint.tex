To prove that there exist infinitely many primes $\equiv {ell} \pmod{{k}}$ we can proceed by contradiction. Suppose there are finitely many primes $\equiv {ell}\pmod{{k}}$ and denote them by $p_1, p_2,\dots,p_m$. Since ${prev_primes_list} \equiv {ell} \pmod{{k}}$, we can write the list as ${prev_primes_list}, p_{{index_prime_unk}},p_{{index_prime_unk_plusone}},\dots, p_m$ (so $p_{{index_prime}}={first_prime}$). {def_Q_alt} Consider the following congruence equation system:
	\begin{equation*}
        \begin{cases}      
		c\equiv {b_value}\pmod{{first_prime}^2}\\
		c\equiv 0\pmod{{k}Q}.
        \end{cases}
	\end{equation*}
The Chinese Remainder Theorem guarantees the existence of $c\in\Z$ that is a solution to the above system since ${first_prime}$ does not divide ${k}Q$. It follows that 
	\begin{gather*}
		\autosizeequation{f(c)\equiv f({b_value})={f_at_b}\equiv {first_prime}\cdot{f_at_b_red}\pmod{{first_prime}^2},}\\
        \autosizeequation{f(c)\equiv f(0)={factor_indep_coef_f}\pmod{{k}Q}.}
	\end{gather*}

In particular, observe that the prime $p_{{index_prime}}={first_prime}$ divides $f(c)$, but ${first_prime}^2$ does not.
\begin{lemma}
Every prime that divides $f(c)$ is $\equiv 1\pmod{{k}}$ $($except for $p_{{index_prime}}={first_prime})$. 
\end{lemma}
\begin{proof}
{begin_lema1_alt} Since $f(c)\equiv {factor_indep_coef_f}\pmod{{k}Q}$ and $r$ is a divisor of ${k}Q$, we deduce that $f(c)\equiv {factor_indep_coef_f}\pmod{r}$. But $r$ is a divisor of $f(c)$, so $f(c)\equiv 0 \pmod{r}$. Therefore, ${factor_indep_coef_f}\equiv 0\pmod{r}$. Thus, it must happen that $r={primes_indep_coef_f_list}$, which {is_are_alt} $\equiv 1 \pmod{{k}}$. This forces $r$ to be $\equiv 1 \pmod{{k}}$, a contradiction.