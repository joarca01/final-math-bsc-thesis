\documentclass[a4paper, 12pt]{article}
\usepackage{geometry}
\geometry{top=2.54cm, bottom=2.54cm, left=2.54cm, right=2.54cm, heightrounded}
\usepackage[utf8]{inputenc}
\usepackage[T1]{fontenc}
\usepackage[english]{babel} 
\usepackage{amsthm, amsmath, amssymb, amsfonts, amscd}
\usepackage{mathtools}
\usepackage{fancyhdr}
\usepackage{setspace}
\usepackage{hyperref}
\usepackage{graphicx}
\usepackage{ifluatex}
\usepackage{array}
\usepackage{multirow}
\usepackage[table]{xcolor}
\usepackage{tcolorbox}
\usepackage{mdframed}

% ---- Spacing ---- %
\linespread{1.0}
\onehalfspacing

% ---- Page header ---- %
\pagestyle{fancy}
\headheight 30pt
\fancyhf{}
\rhead{Page \thepage}
\lhead{\nouppercase{Infinite number of primes in $51n+35$, $n\geqslant 0$}}
\headsep 1.5em

% ---- Link & URL style ---- %
\hypersetup{colorlinks, citecolor=green, linkcolor=blue, urlcolor=blue}
\urlstyle{rm}

% ---- Paragraph indentation style ---- %
\setlength{\parindent}{20pt}

% ---- Box ---- %
\definecolor{framecolor}{rgb}{0.5, 0.5, 0.5}
\definecolor{backcolor}{rgb}{0.94, 0.97, 1.0}
\newtcolorbox{mybox}{colback=backcolor, colframe=framecolor, boxrule=0.5pt, arc=3pt}

% ---- Theorems ---- %
\newtheorem{theorem}{Theorem}%[section]
\newtheorem{lemma}[theorem]{Lemma}

\theoremstyle{definition}
\newtheorem{definition}[theorem]{Definition}

% ---- Definitions ---- %
\renewcommand{\qedsymbol}{\rule{0.7em}{0.7em}} % Black box to end proofs. 
\newcommand{\Z}{\ensuremath{\mathbb{Z}}}
\newcommand{\Q}{\ensuremath{\mathbb{Q}}}
\newcommand{\C}{\ensuremath{\mathbb{C}}}
\newcommand{\F}{\ensuremath{\mathbb{F}}}
\DeclareMathOperator{\degpol}{deg} % Degree of a polynomial.
\DeclareUnicodeCharacter{03B6}{\zeta} % Define Unicode Character 'ζ'. 
\newcommand{\splitatcommas}[1]{%
  \begingroup
  \begingroup\lccode`~=`, \lowercase{\endgroup
    \edef~{\mathchar\the\mathcode`, \penalty0 \noexpand\hspace{0pt plus 1em}}%
  }\mathcode`,="8000 #1%
  \endgroup
}% Break long inline expressions after comma sign
\newsavebox{\equationbox} % Save box for storing the equation
\newcommand{\autosizeequation}[1]{%
	\sbox{\equationbox}{$\displaystyle #1$}% Store the equation in the box
	\ifdim \wd\equationbox > \textwidth % Check if the width exceeds text width
	\resizebox{\textwidth}{!}{$\displaystyle #1$}% Resize if too wide
	\else
	#1 % Otherwise, leave it as is
	\fi
}% Define a custom macro that checks the equation's length and resizes if necessary
\renewcommand\title{\textbf{The arithmetic progression $51n+35$, $n\geqslant0$, contains infinitely many primes.\\ A Euclidean proof}}

\begin{document}
\thispagestyle{plain}
\pagenumbering{arabic}
\sloppy
\begin{center}
{\huge \bfseries \title\par}
\vspace{1cm}
\end{center}

\begin{mybox}
{\Large \textbf{About this document}}
\vspace{0.3cm}\\
\textit{This file has been automatically generated for the user-supplied arithmetic progression. The code behind this document can be found in the url \url{https://github.com/joarca01/final-math-bsc-thesis}, and has been developed as part of a BSc Thesis in Mathematics by Joan Arenillas i Cases at the Autonomous University of Barcelona. The above link also provides full access to the complete Thesis. Please use \href{mailto:joanarenillas01@gmail.com}{\nolinkurl{joanarenillas01@gmail.com}} to report any typo or express any suggestions.}
\end{mybox}
\vspace{0.5cm}

We will prove that the arithmetic progression $\equiv 35 \pmod{51}$ contains infinitely many primes. Equivalently, we will see that there are infinitely many primes of the form $51n+35$, $n\geqslant0$. For this purpose, consider the polynomial 
\begin{equation*}
\binoppenalty=10000 % no line break after \cdot
\mathchardef\plus=\mathcode`+ % save the mathcode
\mathcode`+="8000 % make the plus math active
\thinmuskip=0mu
\begingroup\lccode`~=`+
  \lowercase{\endgroup\def~}{%
    \plus\allowbreak\hspace{0pt plus 2pt}%
}
\hphantom{}
\parbox{1.0\displaywidth}{\centering\linespread{1.1}\selectfont
  \makebox[0pt][r]{$ $}$f(x):=x^{16} + 41564 x^{15} + 809798044 x^{14} + 9817081493948 x^{13} + 82882990874110336 x^{12} + 516745943856254528780 x^{11} + 2461045776942047374624636 x^{10} + 9133213097746587155992893740 x^{9} + 26691858129023247806536115380090 x^{8} + 61635331097275521694392753587100407 x^{7} + 112080993547733232403552911983760867409 x^{6} + 158816450223036109059733286885825093520161 x^{5} + 171905565158061875866107680565415107466929985 x^{4} + 137408314622976892172335483399457480336634284510 x^{3} + 76491541061244145651409658257166611628292501466320 x^{2} + 26494764597591439058927261555967710356172555864587511 x + 4301785361217497374788042324368518937185458391375540801.$
}
\end{equation*}

\section{The main Theorem}\label{sec:mainTh}

To prove that there exist infinitely many primes $\equiv 35 \pmod{51}$ we can proceed by contradiction. Suppose there are finitely many primes $\equiv 35\pmod{51}$ and denote them by $p_1, p_2,\dots,p_m$. Since $137 \equiv 35 \pmod{51}$, we can write the list as $137, p_{2},p_{3},\dots, p_m$ (so $p_{1}=137$). Now, let $t$ be the product of every prime divisor of the discriminant of $f$ (denoted by $\Delta(f)$) different from $137$. Define $Q:= t\cdot p_{2}p_{3}\cdots p_m$. Consider the following congruence equation system:
	\begin{equation*}
        \begin{cases}      
		c\equiv 21\pmod{137^2}\\
		c\equiv 0\pmod{51Q}.
        \end{cases}
	\end{equation*}
The Chinese Remainder Theorem guarantees the existence of $c\in\Z$ that is a solution to the above system since $137$ does not divide $51Q$. It follows that 
	\begin{gather*}
		\autosizeequation{f(c)\equiv f(21)=4893214816490069930571611257450088898981718913368488583\equiv 137\cdot51\pmod{137^2},}\\
        \autosizeequation{f(c)\equiv f(0)=6529 \cdot 6733 \cdot 1716587581 \cdot 8120815069 \cdot 7019850568745870836272543037\pmod{51Q}.}
	\end{gather*}

In particular, observe that the prime $p_{1}=137$ divides $f(c)$, but $137^2$ does not.
\begin{lemma}
Every prime that divides $f(c)$ is $\equiv 1\pmod{51}$ $($except for $p_{1}=137)$. 
\end{lemma}
\begin{proof}
Let $r$ be a prime divisor of $f(c)$ different from $137$. In Section \ref{sec:properties} we will establish that $r$ is a prime divisor of $\Delta(f)$, of $51$ or $r\equiv 1,35\pmod{51}$. For now, we will assume this is true. To reach a contradiction, suppose $r\not\equiv 1\pmod{51}$. Thus, $r\equiv 35\pmod{51}$ or $r$ is a prime divisor of $\Delta(f)$ or $51$, so $r$ divides $51\cdot t\cdot p_{2}p_{3}\cdots p_m=51Q$. Since $f(c)\equiv 6529 \cdot 6733 \cdot 1716587581 \cdot 8120815069 \cdot 7019850568745870836272543037\pmod{51Q}$ and $r$ is a divisor of $51Q$, we deduce that $f(c)\equiv 6529 \cdot 6733 \cdot 1716587581 \cdot 8120815069 \cdot 7019850568745870836272543037\pmod{r}$. But $r$ is a divisor of $f(c)$, so $f(c)\equiv 0 \pmod{r}$. Therefore, $6529 \cdot 6733 \cdot 1716587581 \cdot 8120815069 \cdot 7019850568745870836272543037\equiv 0\pmod{r}$. Thus, it must happen that $r=6529, 6733, 1716587581, 8120815069, 7019850568745870836272543037$, which are $\equiv 1 \pmod{51}$. This forces $r$ to be $\equiv 1 \pmod{51}$, a contradiction. Therefore, $f(c)$ is only divisible by primes $\equiv 1\pmod{51}$ (and by $p_{1}=137$).
\end{proof}

Finally, from the fact that $f(c)$ has every prime divisor $\equiv 1 \pmod{51}$ except for $p_{1}=137$ it follows, mod $51$, that $f(c)=1\cdot1\cdots 1\cdot35=35$ (note that $35$ only appears once because $p_{1}=137\equiv35\pmod{51}$ and the fact that $137^2$ does not divide $f(c)$). However, observe that $f(c)\equiv f(0)\equiv 1\pmod{51}$. This is a contradiction. Therefore, the arithmetic progression $\equiv 35\pmod{51}$ contains infinitely many primes.

\section{Properties of the polynomial \texorpdfstring{$f(x)$}{fx}}\label{sec:properties}
 
To complete the proof of the main Theorem in Section \ref{sec:mainTh} we must justify that every prime divisor $p$ of $f(c)$ either belongs to the finite set 
\begin{equation*}
T:=\{p: p \ \text{is a prime divisor of} \ \Delta(f) \ \text{or is a prime divisor of} \ 51\}
\end{equation*}
or satisfies $p\equiv 1,35\pmod{51}$. To see this, we must first recall the expression of the discriminant of a polynomial. 
\begin{definition}
The discriminant of a monic polynomial $A(x)=x^m+a_{m-1}x^{m-1}+\cdots+a_1x+a_0$ is given, in terms of its roots $\{r_1,r_2,\dots,r_m\}\subset\C$ (not necessarily distinct), by
\begin{equation}\label{eq:discrim}
	\Delta(A)=\prod_{i<j}(r_i-r_j)^2, \quad 1\leqslant i,j\leqslant m.
\end{equation}
\end{definition}

It will be useful to remember that the $51$st cyclotomic polynomial is $\Phi_{51}(x)=x^{32} - x^{31} + x^{29} - x^{28} + x^{26} - x^{25} + x^{23} - x^{22} + x^{20} - x^{19} + x^{17} - x^{16} + x^{15} - x^{13} + x^{12} - x^{10} + x^{9} - x^{7} + x^{6} - x^{4} + x^{3} - x + 1$. We shall also define what a \emph{prime divisor} of a given polynomial is. 
\begin{definition}
Let $A(x)\in\Z[x]$ be a polynomial. We say that a prime number $p$ is a \emph{prime divisor} of $A$ (or simply that $p$ \emph{divides} $A$) if there exists $m\in\Z$ such that $p$ divides $A(m)$.
\end{definition}
With the definition above, we are interested in describing the prime divisors of $f$.

Let's now start the proof. Consider the set $S:=\splitatcommas{\{1, 2, 4, 5, 7, 8, 10, 11, 13, 14, 16, 20, 23, 26, 29, 32\}}$ and the values $h(\zeta^{s}):=(\zeta^{s}-51)(51-\zeta^{35s})$, with $s\in S$ and $\zeta:=e^{2\pi i/{51}}$, a $51$st primitive root of unity (thus a root of $\Phi_{51}(x)$). A simple calculation shows that $f(x)$ can be written as
\begin{equation}\label{eq:fpolynomial}
\binoppenalty=10000 % no line break after \cdot
\mathchardef\plus=\mathcode`+ % save the mathcode
\mathcode`+="8000 % make the plus math active
\thinmuskip=0mu
\begingroup\lccode`~=`+
  \lowercase{\endgroup\def~}{%
    \plus\allowbreak\hspace{0pt plus 2pt}%
}
\hphantom{}
\parbox{1.0\displaywidth}{\centering\linespread{1.1}\selectfont
  \makebox[0pt][r]{$ $}$f(x)=\displaystyle\prod_{s\in S}\big(x-h(\zeta^{s})\big)=x^{16} + 41564 x^{15} + 809798044 x^{14} + 9817081493948 x^{13} + 82882990874110336 x^{12} + 516745943856254528780 x^{11} + 2461045776942047374624636 x^{10} + 9133213097746587155992893740 x^{9} + 26691858129023247806536115380090 x^{8} + 61635331097275521694392753587100407 x^{7} + 112080993547733232403552911983760867409 x^{6} + 158816450223036109059733286885825093520161 x^{5} + 171905565158061875866107680565415107466929985 x^{4} + 137408314622976892172335483399457480336634284510 x^{3} + 76491541061244145651409658257166611628292501466320 x^{2} + 26494764597591439058927261555967710356172555864587511 x + 4301785361217497374788042324368518937185458391375540801.$
}
\end{equation}


Now, suppose that $p$ is a prime divisor of $f$ such that $p$ does not belong to $T$. Next, consider a field $\F$ containing both the finite field $\F_p$ and $\zeta$\footnote{For instance, consider $\F=\F_{p^n}$ with a suitable integer $n\geqslant 1$ such that $\Phi_{51}$ has a root $\zeta$.}. Since $p$ divides $f$, working in $\F$, there exists $a\in\Z$ such that 
\begin{equation*}
f(a)=\prod_{s'\in S}\big(a-h(\zeta^{s'})\big)=0.
\end{equation*}
Since $\F$ is a field, there exists some $s\in S$ such that $a=h(\zeta^{s})$.
\begin{lemma}
The equality $h(\zeta^s)=h(\zeta^{ps})$ holds in $\F$.
\end{lemma}
\begin{proof}
 Observe that the following calculation holds in $\F$:
\begin{align}\label{eq:reproots}
h(\zeta^{s})&=a\notag \\ 
&=a^p\notag \\ 
&=h(\zeta^{s})^p\notag \\ 
&=(\zeta^{s}-51)^p(51-\zeta^{35s})^p\notag \\ 
&=(\zeta^{ps}-51^p)(51^p-\zeta^{35ps})\notag \\ 
&=(\zeta^{ps}-51)(51-\zeta^{35ps})\notag \\ 
&=h(\zeta^{ps}),
\end{align}
where we have used Fermat little theorem in the second equality. The fifth equality, on the other hand, relies on the fact that $\F$ has characteristic $p$ (so that $(c+d)^p=c^p+d^p$ for every $c,d \in \F$) and the following one, on Fermat little theorem.
\end{proof}

Therefore, equality \eqref{eq:reproots} means that $h(\zeta^{ps})=h(\zeta^{s})$ is a root of $\overline{f(x)}\in\F[x]$. 

\begin{lemma}
$h(\zeta^{ps})$ is also a root of $f(x)$ in $\Q(\zeta)$ $($the smallest subfield of $\C$ containing $\zeta)$.
\end{lemma}
\begin{proof}
Begin by noting that the value $h(\zeta^{ps})$ only depends on the value of $ps \bmod{51}$ since it only appears as an exponent of $\zeta$. Since $p$ does not divide $51$ and $s$ is coprime to $51$, $ps$ is coprime to $51$ (so $ps \bmod{51}$ is coprime to $51$) and hence $\zeta^{ps}$ is a primitive $51$st root of unity.

There are now only two options: either $ps \bmod{51}\in S$ or $ps \bmod{51}\notin S$. In the first case, $h(\zeta^{ps})$ is a root of $f(x)$, observing expression \eqref{eq:fpolynomial}. In the latter case, note that every integer $ps \bmod{51}$ relatively prime to $51$ not in $S$ satisfies $ps\equiv 35t\pmod{51}$ for some $t\in S$ (for instance, if $ps \bmod{51}=49$, pick $t=32\in S$ so that $49\equiv 35\cdot32 \pmod{51}$). This means that $h(\zeta^{ps})=h(\zeta^{35t})$. Let us prove that $h(\zeta^{35t})=h(\zeta^{t})$, so $h(\zeta^{ps})=h(\zeta^{35t})=h(\zeta^{t})$ is also a root of $f(x)$. Indeed,
\begin{align*}
h(\zeta^{35t})&=(\zeta^{35t}-51)(51-\zeta^{35^2t})\\
&=(\zeta^{35^2t}-51)(51-\zeta^{35t})\\
&=(\zeta^{t}-51)(51-\zeta^{35t})\\
&=h(\zeta^{t}),
\end{align*}
where we have used that $\zeta^{35^2t}$ only depends on the value of $35^2t \bmod{51}$ and the fact that $35^2\equiv 1\pmod{51}$. Therefore, $h(\zeta^{ps})=h(\zeta^{t})$ is always a root of $f(x)$ in $\Q(\zeta)$.
\end{proof}

\begin{lemma}
$h(\zeta^{ps})$ and $h(\zeta^{s})$ are the same root of $f(x)$ in $\Q(\zeta)$.
\end{lemma}
\begin{proof}
If $h(\zeta^{ps})$ and $h(\zeta^{s})$ were two distinct roots of $f(x)$ in $\Q(\zeta)$, we know because of \eqref{eq:reproots} that they would be the same in $\F$. Therefore, observing expression \eqref{eq:discrim}, it follows that $\Delta(f \bmod{p})=\Delta(f) \bmod{p}=0$, so $p$ divides $\Delta(f)$. This is a contradiction with our choice of $p$. Thus, $h(\zeta^{ps})$ and $h(\zeta^{s})$ are in fact the same root of $f(x)$ in $\Q(\zeta)$.
\end{proof}

Therefore, the equality
\begin{equation*}
h(\zeta^{ps})=\big(\zeta^{ps}-51\big)\big({51}-\zeta^{35ps}\big)=\big(\zeta^{s}-51\big)\big({51}-\zeta^{35s}\big)=h(\zeta^{s})
\end{equation*}
holds in $\Q(\zeta)$. Next, write the above equation in terms of $\theta:=\zeta^{s}$ and multiply both sides by $-1$. This changes yield
\begin{align}
2601-{51}\big(\theta^{p}+\theta^{35p}\big)+\theta^{(1+35)p}&=2601-{51}\big(\theta+\theta^{35}\big)+\theta^{1+35}, \notag \\
-51\big(\theta^{p}+\theta^{35p}\big)+\theta^{36p}&=-51\big(\theta+\theta^{35}\big)+\theta^{36}\label{eq:equality_in_theta}.
\end{align}
The right-hand side of the equation above does not depend on $p$. The left-hand side only depends on the value of $p\bmod{51}$, since $p$ only appears as an exponent of $\theta$ (and $\theta$ is a primitive $51$st root of unity). The above equality gives information about $p$, which is what we are interested in. 

\begin{lemma}
The fact that \eqref{eq:equality_in_theta} holds implies that $p \bmod{51}\in H:=\{1,35\}$.
\end{lemma}
\begin{proof}
To prove this, we will check every value of $p$ such that $p \bmod{51}\notin H$ and conclude that \eqref{eq:equality_in_theta} is not true in $\Q(\theta)$ in those cases. Therefore, we shall see the following: if $p \bmod{51}\in G\setminus H=\splitatcommas{\{2, 4, 5, 7, 8, 10, 11, 13, 14, 16, 19, 20, 22, 23, 25, 26, 28, 29, 31, 32, 37, 38, 40, 41, 43, 44, 46, 47, 49, 50\}}$, then $-h(\theta^p)\neq -h(\theta)$. This will automatically imply what we want to prove: since \eqref{eq:equality_in_theta} holds, $p \bmod{51}\in H$. To see this, rewrite \eqref{eq:equality_in_theta} as
\begin{equation}\label{eq:equalityQ}
-51\big(\theta^{p}+\theta^{35p}\big)+\theta^{36p}+51\big(\theta+\theta^{35}\big)-\theta^{36}=0
\end{equation}
and trade $\theta$ for $x$, since the condition \eqref{eq:equalityQ} in $\Q(\theta)$ is equivalent to the condition
\begin{equation}\label{eq:polycheck0}
-51\big(x^p+x^{35p}\big)+x^{36p}+51\big(x+x^{35}\big)-x^{36}=0
\end{equation}
in $\Q[x]/(\Phi_{51}(x))\cong\Q(\theta)$ ($\Phi_{51}(x)$ is also the minimal polynomial of $\theta$). We will explicitly write the case $p=2 \bmod{51}$ (the remaining values of $p \bmod{51}\in G\setminus H$ are left as an exercise to the reader). With this value of $p$, equation \eqref{eq:polycheck0} becomes
\begin{align*}
A(x)&:=-51\big(x^{2}+x^{70}\big)+x^{72}+51\big(x+x^{35}\big)-x^{36}\\
&=x^{72} - 51 x^{70} - x^{36} + 51 x^{35} - 51 x^{2} + 51 x=0.
\end{align*}

If we recall that $\Q(\theta)\cong\Q[x]/(\Phi_{51}(x))$, the above equation is equivalent to $A(x)$ being a multiple of the $51$st cyclotomic polynomial, $\Phi_{51}(x) = x^{32} - x^{31} + x^{29} - x^{28} + x^{26} - x^{25} + x^{23} - x^{22} + x^{20} - x^{19} + x^{17} - x^{16} + x^{15} - x^{13} + x^{12} - x^{10} + x^{9} - x^{7} + x^{6} - x^{4} + x^{3} - x + 1$. Therefore, we are interested in showing that the residue $R(x)$ of the division $A(x)/\Phi_{51}(x)$ satisfies $R(x)\neq 0$, from which our result will follow. A simple Euclidean division of polynomials shows that $A(x)=B(x)\cdot\Phi_{51}(x)+(x^{21} - 50 x^{19} - 51 x^{18} - 50 x^{2})$, with $B(x)$ a polynomial of degree $40$, so $R(x)=x^{21} - 50 x^{19} - 51 x^{18} - 50 x^{2} \neq 0$. Therefore, equality \eqref{eq:equality_in_theta} implies that $p \bmod{51} \in H$, that is, $p\equiv 1,35\pmod{51}$.
\end{proof}

In conclusion, every prime divisor $p$ of $f(c)$ either belongs to the finite set 
\begin{equation*}
T:=\{p: p \ \text{is a prime divisor of} \ \Delta(f) \ \text{or is a prime divisor of} \ 51\}
\end{equation*}
or satisfies $p\equiv 1,35\pmod{51}$, which finally settles the main Theorem in Section \ref{sec:mainTh}. 

\end{document}