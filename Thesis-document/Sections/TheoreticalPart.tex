\documentclass[../main.tex]{subfiles}

\begin{document}
\section{The scope of Euclidean proofs}\label{sec:theoretical}

Our goal is to show that one can find Euclidean proofs of the infinitude of primes $\equiv\ell\pmod{k}$ if and only if $\ell^2\equiv 1 \pmod{k}$. This section will primarily be an expanded and upgraded version of the main propositions and theorems in \cite{Murty} and \cite{Conrad}.

Let us fix the notation. In this section, $k$ and $\ell$ will again denote a fixed pair of non-zero positive integers, which will univocally identify the arithmetic progression $kn+\ell$, $n\geqslant 0$, that is, integers $\equiv \ell\pmod{k}$. We will always suppose that they are relatively prime to satisfy Dirichlet \cref{th:Dirichlet} and we may additionally suppose that $k > \ell$ and $k\neq 1,2$.

\begin{remark}
	The cases $k=1,2$ are highly degenerate, and a Euclidean proof can be easily established (see \cref{sec:smallcases} in the Appendix).
\end{remark}

\subsection{Schur Theorem}\label{sec:SchurTh}

We will start by proving that one can find a Euclidean proof of the infinitude of primes $\equiv \ell\pmod{k}$ if $\ell^2\equiv 1\pmod{k}$, following Schur's proof detailed in \cite{Murty}. Let $\zeta$ be a $k$th primitive root of unity (take $\zeta=e^{2\pi i/k}$) and let $K:=\Q(\zeta)$. We know because of \cref{prop:Galisominteg} that $\Gal(K/\Q)\cong (\Z/k\Z)^\times=G$, the group of coprime residue classes modulo $k$. Thus, the number of elements in $G$ is $\varphi(k)$ and we may identify the field automorphism $\sigma_i\in\Gal(K/\Q)$ sending $\zeta\mapsto\zeta^i$ with the integer $i\in G$.

Consider the subgroup $H:=\{1, \ell\}\leqslant G$, where for now we may suppose $\ell\not\equiv 1 \pmod{k}$\footnote{The case $\ell\equiv1\pmod{k}$ is easier and will be proved in \cref{sec:Schur_l_cong1}.}. (The proof of the following results for a general subgroup $H$ is given in \cite{Murty}). Since $G$ is finite, the coset representatives\footnote{Since the elements of $G$ are residue classes, the coset representatives of the subgroup $H$ in $G$ are also residue classes.} of $H$ in $G$ form a finite set $S$. Observe that $\abs{S}=[G:H]=\abs{G}/\abs{H}=\varphi(k)/2$, so one can write 
\begin{equation}\label{eq:GsplitH}
	G=\bigsqcup_{s\in S}sH=\bigsqcup_{s\in S}\{s, s\ell\}.
\end{equation}
Note that $\abs{S}$ is well-defined, since the only cases when $\varphi(k)$ is odd happen for $k=1, 2$, but we are always excluding these cases. Also, we can always suppose that $1$ lies in $S$.

Now define the polynomial
\begin{equation*}
	h_u(z):=(z-u)(u-z^{\ell})\in\Z[z], 
\end{equation*}
which depends on some $u\in\Z$. Observe the following lemma.
\begin{lemma}\label{lemma:propertiesh1}
	 The equality $h_u(\zeta)^s=h_u(\zeta^s)$ holds for every $s\in S$.
\end{lemma}
\begin{proof}
Note that $s$ is coprime to $k$, so $\sigma_s$ belongs to $\Gal(K/\Q)$. Then,
\begin{align*}
	h_u(\zeta)^s&=\sigma_s(h_u(\zeta))=\sigma_s\big((\zeta-u)(u-\zeta^\ell)\big)\\
	&=(\sigma_s(\zeta)-u)(u-\sigma_s(\zeta^\ell))=(\zeta^s-u)(u-\zeta^{s\ell})=h_u(\zeta^s),
\end{align*}
where in the third equality we used that $\sigma_s$ is a field automorphism (hence multiplicative and additive) that fixes the elements in $\Q$. 
\end{proof}

Set $\eta:=h_u(\zeta)=(\zeta-u)(u-\zeta^{\ell})\in\C$. The previous result leads to the following result:
\begin{lemma}\label{lemma:propertiesh}
	Let $a\in G$. Then $\sigma_a(\eta)=\sigma_s(\eta)$, where $s$ is the coset representative of $a$.
\end{lemma}
\begin{proof}
	Since $a$ belongs to $G$, it must happen that $a=s$ or $a=s\ell$ for some coset representative $s\in S$, because of \cref{eq:GsplitH}. In the first case, there is nothing to prove, so suppose $a=s\ell$. Now,
	\begin{align*}
		\sigma_a(\eta)&=\sigma_{s\ell}(\eta)=\sigma_{s\ell}\big((\zeta-u)(u-\zeta^\ell)\big)=(\sigma_{s\ell}(\zeta)-u)(u-\sigma_{s\ell}(\zeta^\ell))\\
		&=(\zeta^{s\ell}-u)(u-\zeta^{s\ell^2})=(u-\zeta^{s})(\zeta^{s\ell}-u)=(\zeta^{s}-u)(u-\zeta^{s\ell})=h_u(\zeta^s),
	\end{align*}
	where again we used that $\sigma_{s\ell}$ is a field automorphism in the third equality. In the fifth equality we used that $\zeta^{s\ell^2}$ only depends on the value of $s\ell^2 \bmod{k}$. Since $\ell^2\equiv 1 \pmod{k}$, it follows that $\zeta^{s\ell^2}=\zeta^s$. Finally, because of \cref{lemma:propertiesh1}, 
	\begin{equation*}
		\sigma_a(\eta)=h_u(\zeta^s)=h_u(\zeta)^s=\sigma_s(\eta).
	\end{equation*}
\end{proof}

Observe that $\zeta$ (and $\zeta^s$, $s\in S$) are algebraic integers, and $h_u$ has integer coefficients, so $h_u(\zeta^s)$ are also algebraic integers. Consider the monic polynomial
\begin{equation}\label{eq:polinomi}
	f_u(x):=\prod_{s\in S}\big(x-h_u(\zeta^s)\big)=\prod_{s\in S}\big(x-(\zeta^s-u)(u-\zeta^{\ell s})\big),
\end{equation}
whose coefficients are symmetric polynomials in the algebraic integers $h_u(\zeta^s)$, and so they are algebraic integers themselves. Since $\sigma\in\Gal(\overline{\Q}/\Q)$ permutes the $k$th primitive roots of unity ($\overline{\Q}$ is the algebraic closure of $\Q$), $\sigma(h_u(\zeta^s))=h_u(\sigma(\zeta)^s)$ is another root of $f_u$, so $\sigma$ permutes the roots of $f_u$. Since the coefficients of $f_u$ are symmetric polynomials on $h_u(\zeta^s)$, these coefficients are fixed by every $\sigma\in\Gal(\overline{\Q}/\Q)$, so $\sigma(f_u)=f_u$. Thus, the coefficients of $f_u$ lie in $\Q$, and they must be integers since the only algebraic integers in $\Q$ are the integers, so $f_u$ belongs to $\Z[x]$. Moreover, this polynomial is irreducible.
\begin{proposition}\label{prop:fuirreducible}
	The field $L:=\Q(\eta)$ is the fixed field of $H$, except for finitely many values of $u$. Excluding these cases, the polynomial $f_u$ is irreducible.
\end{proposition}
\begin{proof}
	Consider the tower of extensions $K/L=\Q(\zeta)^H/\Q$, where we have defined $L$ to be the fixed field of $H$. If $k=3,4,6$, then $\eta$ is an integer and the fixed field of $H$ is just $\Q$ since $[K^H:\Q]=\varphi(k)/2=1$, due to the Galois correspondence. In these cases, $f_u$ is irreducible since it has degree $1$ because $\abs{S}=1$. Thus, suppose $k$ is such that $\varphi(k)>2$, that is, $k=5$ or $k\geqslant 7$. Thanks to the Galois correspondence, we want to see that, except for finitely many values of $u$, $\sigma(\eta)=\eta$ if and only if $\sigma$ belongs to $H$, for every $\sigma\in G$. 
	
	The converse implication is simple. Denote by $\sigma_\ell$ the only non-trivial automorphism in $H$. Since the coset representative of $\ell$ is $s=1$, it follows from \cref{lemma:propertiesh} that $\sigma_\ell(\eta)=\sigma_{s=1}(\eta)=\eta$.
	
	We now have to show that the equality $\sigma(\eta)=\eta$ implies that $\sigma$ belongs to $H$. We will instead prove the equivalent contrapositive assertion: if $\sigma\in\Gal(K/\Q)$ does not belong to $H$, then $\sigma(\eta)\neq\eta$. Because of \cref{lemma:propertiesh}, it is enough to check it for the automorphisms $\sigma_s$ with $s\neq 1$, which excludes $\sigma_1$ and $\sigma_\ell$ (the automorphisms in $H$). From \cref{lemma:propertiesh1}, we have
	\begin{equation*}
		\sigma_s(\eta)=\eta^s=h_u(\zeta)^s=h_u(\zeta^s).
	\end{equation*}
	Thus, we have to prove that $h_u(\zeta^s)\neq \eta$, for every $s\in S^*:=S\setminus \{1\}$. 
	
	Note that $\eta=h_u(\zeta)=(\zeta-u)(u-\zeta^\ell)$ can be interpreted as a polynomial in $u$. Now observe that, for every $s\in S^*$, there are only a finite number of values of $u\in\Z$ for which the equality of polynomials in $u$
	\begin{equation}\label{eq:equalityetas}
		h_u(\zeta^s)=(\zeta^s-u)(u-\zeta^{\ell s})=(\zeta-u)(u-\zeta^\ell)=\eta
	\end{equation}
	holds, so we may exclude these integers and conclude that $\Q(\eta)$ is the fixed field of $H$. However, one may worry about the fact that 
	\begin{align}
		h_u(\zeta^s)&=\eta \notag,\\
		u^2-(\zeta^s+\zeta^{\ell s})u+\zeta^{s(1+\ell)}&=u^2-(\zeta+\zeta^\ell)u+\zeta^{1+\ell} \notag,\\
		-(\zeta^s+\zeta^{\ell s})u+\zeta^{s(1+\ell)}&=-(\zeta+\zeta^\ell)u+\zeta^{1+\ell}\label{eq:infinitesolsu}
	\end{align}
	can in principle have infinite solutions for $u$ if $\zeta^s+\zeta^{\ell s}=\zeta+\zeta^\ell$ and $\zeta^{s(1+\ell)}=\zeta^{1+\ell}$. We will show this is never the case. Observe that \cref{eq:infinitesolsu} would have infinitely many solutions for $u$ if and only if $\zeta+\zeta^\ell$ and $\zeta^{1+\ell}$ are fixed by $\sigma_s$. In this case, the equality $\zeta^{1+\ell}=\zeta^{s(1+\ell)}$ tells us that $1+\ell\equiv s(1+\ell)\pmod{k}$. Rearranging terms, we get $(1-s)(1+\ell)\equiv 0\pmod{k}$. Since $s\not\equiv 1\pmod{k}$, then $1-s\not\equiv 0\pmod{k}$, so $1-s$ is invertible modulo $k$. This immediately tells us that $1+\ell\equiv 0\pmod{k}$, that is, $\ell\equiv -1\pmod{k}$. Hence, $\zeta+\zeta^{-1}$ is fixed by $\sigma_s$. But $\zeta+\zeta^{-1}=\zeta+\bar{\zeta}=2\Re(\zeta)=2\cos(2\pi/k)$ generates the totally real subfield of $K$. 
	
	Indeed, observe that complex conjugation $\tau:\zeta\mapsto\bar{\zeta}=\zeta^{-1}$ belongs to $G$ and has order $2$, so the fixed field of $\{\id,\tau\}$ is the maximal real subfield of $K$, which is of degree $2$ below $K$. Also, this field must coincide with the real field $\Q(\zeta+\zeta^{-1})\subset K$, which is also fixed by $\{\id,\tau\}$ and is of degree $2$, since $(x-\zeta)(x-\zeta^{-1})=x^2-(\zeta+\zeta^{-1})x+1$ is the minimal polynomial of $\zeta$ over $\Q(\zeta+\zeta^{-1})$. Since this field has degree $2$, by the Galois correspondence it must be the fixed field of the group $\{1,-1\}=\{1,\ell\}$, so $\sigma_s$ belongs to this group. Then, $s=1$, but this does not happen since we chose the coset $s$ to lie in $S^*$. Thus, \cref{eq:equalityetas} only has a finite number of solutions for $u$. Excluding these finite values of $u$, we have that $\sigma_s(\eta)\neq \eta$ for $s\in S^*$, so finally $L=\Q(\eta)$.
	
	To prove the last part of the proposition, take some $u$ such that $L=\Q(\eta)$. We will now show that every $\eta^s=h_u(\zeta^s)$, for $s\in S$, is different. To reach a contradiction, suppose $h_u(\zeta^{s_1})=h_u(\zeta^{s_2})$ for $s_1\neq s_2$. We can write $\sigma_{s_1}(h_u(\zeta))=\sigma_{s_2}(h_u(\zeta))$, because of \cref{lemma:propertiesh1}. Thus, in terms of $\eta$, we have $\sigma_{s_2}^{-1}\sigma_{s_1}(\eta)=\eta$. Defining $\phi:=\sigma_{s_2}^{-1}\sigma_{s_1}$, then $\phi$ fixes $L$, so $\phi$ belongs to $\Gal(K/L)=H$, which implies that $\sigma_{s_1}H=\sigma_{s_2}H$. That is, $s_1$ and $s_2$ are in the same coset of $H$, which is a contradiction. Therefore, $\eta^s$ for $s\in S$ (which are the roots of $f_u$) are all distinct. This guarantees that $f_u\in\Z[x]$ is separable. 

	 Now, $L=K^H/\Q$ is a Galois extension from the fact that $H$ is normal on $G$ (see \cref{th:galoisextcondition}), because every subgroup of an abelian group is normal. Thus, every automorphism in $\Gal(L/\Q)$ descends from some $\sigma$ in $\Gal(K/\Q)$ due to the Fundamental Theorem of Galois theory. Hence, $\Gal(L/\Q)$ also acts transitively on the roots of $f_u$. Also, $L$ is the splitting field of $f_u$ since its roots are $\eta^s$ for $s\in S$ and each $\eta^s$ lies in $L$. Thus, from \cref{th:conjugatesirreduc}, we have that $f_u$ is a (monic) irreducible polynomial over $\Q[x]$.
\end{proof}

In the following, we will suppose that $u\in \Z$ is always chosen so that $L$ is the fixed field of $H$.

\begin{remark}\label{remark:minpoleta}
Some remarks can be made in relation to the tower of extensions $K/L/\Q$. In light of the Galois correspondence, we have that $[L=K^H:\Q]=[G:H]=\varphi(k)/2$, and also $[K:K^H]=\abs{H}=2$. Similarly, $[K:\Q]=\abs{G}=\varphi(k)$. Also, $f_u$ is the minimal polynomial of $\eta$, since it is irreducible, $f_u(\eta)=0$, and $\degpol(f_u)=\abs{S}=\varphi(k)/2$. In short, $f_u$ is a polynomial of degree $\varphi(k)/2$ whose roots generate $L$ and are invariant under the action of the subgroup $H$. Indeed,
 \begin{align*}
 	\sigma_\ell(h_u(\zeta^s))&=(\zeta^{s\ell}-u)(u-\zeta^{s\ell^2})=(\zeta^{s\ell}-u)(u-\zeta^{s})\\
 	&=(\zeta^{s}-u)(u-\zeta^{s\ell})=h_u(\zeta^s),
 \end{align*}
 since $\ell^2\equiv 1 \pmod{k}$.
\end{remark}

One more remark is needed before moving on to the next theorem. From now on, $p$ will always denote a prime number.

\begin{remark}\label{rem:Ffield}
	For the following results we will need to consider a field $\F$ containing both the finite field $\F_p$ and $\zeta$. For instance, consider $\F=\F_{p^n}$ with a suitable integer $n\geqslant 1$ such that $\Phi_k$ has a root $\zeta$. This is possible because $\F_p\subseteq\F_{p^n}\subseteq\overline{\F_p}$, where $\overline{\F_p}$ is the algebraic closure of $\F_p$. Obviously, in this context we cannot think of $\zeta$ as an element of $\C$, but rather as some root of an irreducible factor of $\overline{\Phi_k}\in\F_p[x]$ over $\F_p$. Alternatively, this field $\F$ can be constructed via $\F_p[x]/(r(x))$ for some factor $r(x)$ of $\Phi_k$ that is irreducible over $\F_p$.
\end{remark}

The following theorem will play a pivotal role in this thesis.
\begin{theorem}[\textbf{Schur}]\label{th:fdivisors}
	Every prime divisor of $f_u$ belongs to the residue classes of $H$ (except for finitely many prime divisors).
\end{theorem}
\begin{proof}
	  Let $T$ be the set containing every prime that divides $k$ or $\Delta(f_u)$, where $\Delta(f_u)$ is the discriminant of $f_u$. Note that $T$ is finite, due to the Fundamental Theorem of Arithmetic. Now, take a prime divisor $p$ of $f_u$ such that $p$ does not lie in $T$.
	  
	  Since $p$ divides $f_u$, working in the field $\F$ of \cref{rem:Ffield}, there exists $a\in\Z$ such that 
	  \begin{equation*}
	  	f_u(a)=\prod_{s'\in S}\big(a-h_u(\zeta^{s'})\big)=0.
	  \end{equation*}
	  Since $\F$ is a field, there exists some $s\in S$ such that $a=h_u(\zeta^{s})$. We will now prove that the equality $h_u(\zeta^s)=h_u(\zeta^{ps})$ holds in $\F$. Observe that in this field
	  \begin{align}\label{eq:reproots}
	  	h_u(\zeta^{s})&=a=a^p=h_u(\zeta^{s})^p=(\zeta^{s}-u)^p(u-\zeta^{\ell s})^p\notag\\ 
	  	&=(\zeta^{ps}-u^p)(u^p-\zeta^{\ell ps})=(\zeta^{ps}-u)(u-\zeta^{\ell ps})=h_u(\zeta^{ps}),
	  \end{align}
	  where we have used Fermat little theorem in the second equality. The fifth equality, on the other hand, relies on the fact that $\ch(\F)=p$ (so that $(c+d)^p=c^p+d^p$ for every $c,d \in \F$) and the following one, on Fermat little theorem. Therefore, equality \cref{eq:reproots} means that $h_u(\zeta^{ps})=h_u(\zeta^{s})$ is a root of $\overline{f_u}\in\F[x]$.
	  
	  We will now see that $h_u(\zeta^{ps})$ is also a root of $f_u$ in $K$. Begin by noting that the value $h_u(\zeta^{ps})$ only depends on the value of $ps \bmod{k}$ since it only appears as an exponent of $\zeta$. Since $p$ does not divide $k$ by hypothesis and $s$ is coprime to $k$, $ps$ is coprime to $k$ (so $ps \bmod{k}$ is coprime to $k$) and hence $\zeta^{ps}$ is a primitive $k$th root of unity.
	  
	  There are now only two options: either $ps \bmod{k}$ belongs to $S$ or $ps \bmod{k}$ does not belong to $S$. In the first case, $h_u(\zeta^{ps})$ is a root of $f_u$ in $K$, observing expression \cref{eq:polinomi}. In the latter case, note that every integer $ps \bmod{k}$ relatively prime to $k$ not in $S$ satisfies $ps\equiv \ell t\pmod{k}$ for some $t\in S$ (because of \cref{eq:GsplitH}). This means that $h_u(\zeta^{ps})=h_u(\zeta^{\ell t})$. Let us prove that $h_u(\zeta^{\ell t})=h_u(\zeta^{t})$, so $h_u(\zeta^{ps})=h_u(\zeta^{\ell t})=h_u(\zeta^{t})$ is also a root of $f_u$ in $K$. Indeed,
	  \begin{align*}
	  	h_u(\zeta^{\ell t})&=(\zeta^{\ell t}-u)(u-\zeta^{\ell^2t})=(\zeta^{\ell^2t}-u)(u-\zeta^{\ell t})\\
	  	&=(\zeta^{t}-u)(u-\zeta^{\ell t})=h_u(\zeta^{t}),
	  \end{align*}
	  where we have used that $\zeta^{\ell^2t}$ only depends on the value of $\ell^2t \bmod{k}$ and the fact that $\ell^2\equiv 1\pmod{k}$. Therefore, $h_u(\zeta^{ps})=h_u(\zeta^{t})$ is always a root of $f_u$ in $K$.
	  
	  We will now see that $h_u(\zeta^{ps})$ and $h_u(\zeta^{s})$ are the same root of $f_u$ in $K$. If $h_u(\zeta^{ps})$ and $h_u(\zeta^{s})$ were two distinct roots of $f_u$ in $K$, we know because of \cref{eq:reproots} that they would be the same in $\F$. Therefore, observing the discriminant expression in \cref{eq:discrim}, it follows that $\Delta(f_u \bmod{p})=\Delta(f_u) \bmod{p}=0$, so $p$ divides $\Delta(f_u)$. This is a contradiction with our choice of $p$. Thus, $h_u(\zeta^{ps})$ and $h_u(\zeta^{s})$ are in fact the same root of $f_u$ in $K$, so $h_u(\zeta^{ps})=h_u(\zeta^{s})$.
	  
	  Since $p$ does not lie in $T$, it follows that $\gcd(k,p)=1$, so one can consider the field automorphism $\sigma_p\in\Gal(K/\Q)$. Following the spirit of \cref{lemma:propertiesh1}, one can easily see that $h_u(\zeta^{ps})=h_u(\zeta^s)^p$, so $h_u(\zeta^s)^p=h_u(\zeta^{ps})=h_u(\zeta^{s})$ in $K$. Therefore, $\eta^s=h_u(\zeta^{s})$ is fixed by $\sigma_p$ and so is $\Q(\eta^s)$. Now, $\Q(\eta^s)=L$, because $\eta^s$ is a conjugate of $\eta$ (thus, $\Q(\eta^s)/\Q$ is Galois since $L/\Q$ is Galois). Consequently, $\sigma_p$ also fixes $L$, and since  $L=K^H$ by definition, it must happen that $p \bmod{k}$ belongs to $H$, that is, $p\equiv 1,\ell \pmod{k}$.
\end{proof}

The special thing about $f_u$ is that we can ``control'' its prime divisors: if $p$ is a prime divisor of $f_u$, then either $p$ divides $k$, or $p$ divides $\Delta(f_u)$ or $p\equiv 1, \ell \pmod{k}$. Nevertheless, we cannot yet guarantee that $f_u$ is a Euclidean polynomial: it satisfies every condition in \cref{def:Euclideanproof}, except that we do not know whether it has infinitely many prime divisors $\equiv\ell\pmod{k}$. This is resolved with the following proposition, which is the converse of the previous theorem.

\begin{proposition}\label{prop:fdivisorsconverse}
	Any prime belonging to any residue class of $H$ divides $f_u$.
\end{proposition}
\begin{proof}
	Let $p$ be a prime belonging to a residue class of $H$. That means $p \bmod{k}$ belongs to $H$. By definition, $\eta=h_u(\zeta)$. Now, since $\gcd(k, p)=1$ and $1\in H$ is the coset representative of $p \bmod{k}$, it holds that
	\begin{equation}\label{eq:solxp}
		\eta^p=\sigma_p(\eta)=\sigma_{s=1}(\eta)=\eta,
	\end{equation}
	because of \cref{lemma:propertiesh}.
	
	Consider now the equation $x^p-x$ and let us work in the field $\F$ of \cref{rem:Ffield}. Since $\F$ is a field with $\ch(\F)=p$ there are $p$ solutions to this equation, since $x$ lies in $\F$ if and only if $x^p=x$. In fact, there exist exactly $p$ integer solutions to the equation: thanks to Fermat little theorem, $0,\dots,p-1$ are roots of $x^p-x \bmod{p}$. Since $\eta^p-\eta=0$ because of \cref{eq:solxp}, $\eta$ is also a solution, and it must be an integer, so $\eta=b$ for some integer $b$. It is now enough to recall that $\eta$ is a root of $f_u$ to conclude the proof. Indeed, the equality $0=f_u(\eta)=f_u(b)$ holds in $\F[x]$. Since $\ch(\F)=p$, $p$ divides $f_u(b)$, and so $p$ is a prime divisor of $f_u$.
\end{proof}

Since there exist infinitely many primes $\equiv\ell\pmod{k}$, $f_u$ has infinitely many prime divisors of this type, and we can finally establish that $f_u$ is a Euclidean polynomial, which will be used in our Euclidean proof.

\begin{remark}\label{remark:trueprimedivf}
	Observe that \cref{th:fdivisors} tells us that, except for finitely many primes, we have $\Spl_1(f_u)\subseteq\{p: p\equiv 1,\ell\pmod{k}\}$. With this last \cref{prop:fdivisorsconverse} we have that $\{p: p\equiv 1,\ell\pmod{k}\}\subseteq\Spl_1(f_u)$. Thus, $\Spl_1(f_u)=\{p: p\equiv 1,\ell\pmod{k}\}$, except for finitely many primes.
\end{remark}

We will also need the following result.

\begin{proposition}[\textbf{Schur}]\label{prop:fdivisors_l_cong1}
	Every prime divisor of $\Phi_k$ not dividing $k$ is $\equiv 1 \pmod{k}$. 
\end{proposition}
\begin{proof}
	Let $T$ be the set containing every prime that divides $k$ or $\Delta(\Phi_k)$. Recall that the primes that divide $\Delta(\Phi_k)$ also divide $k$ (see \cref{prop:discprimediv}) so $T$ effectively contains the prime divisors of $k$. Note that $T$ is finite, due to the Fundamental Theorem of Arithmetic. Now, consider a prime divisor $p$ of $\Phi_k$ such that $p$ does not lie in $T$.
	
	Since $p$ divides $\Phi_k$, working in the field $\F$ of \cref{rem:Ffield}, there exists $a\in\Z$ such that 
	\begin{equation*}
		\Phi_k(a)=\prod_{s'\in S}\big(a-\zeta^{s'}\big)=0.
	\end{equation*}
	Since $\F$ is a field, there exists some $s\in S$ such that $a=\zeta^{s}$. We will now prove that the equality $\zeta^s=\zeta^{ps}$ holds in $\F$. Observe that in this field
	\begin{equation}\label{eq:reproots2}
		\zeta^{s}=a=a^p=\zeta^{ps},
	\end{equation}
	where we have used Fermat little theorem in the second equality. Therefore, equality \cref{eq:reproots2} means that $\zeta^{ps}=\zeta^{s}$ is a root of $\overline{\Phi_k}\in\F[x]$. 
	
	We will now show that $\zeta^{ps}$ is also a root of $\Phi_k$ in $K$. Begin by noting that the value $\zeta^{ps}$ only depends on the value of $ps \bmod{k}$ since it only appears as an exponent of $\zeta$. Since $p$ does not divide $k$ by hypothesis and $s$ is coprime to $k$, $ps$ is coprime to $k$ (so $ps \bmod{k}$ is coprime to $k$), and hence $\zeta^{ps}$ is a primitive $k$th root of unity. Thus, $\zeta^{ps}$ is a root of $\Phi_k$ in $K$.
	
	We will now show that $\zeta^{ps}$ and $\zeta^{s}$ are the same root of $\Phi_k$ in $K$. If $\zeta^{ps}$ and $\zeta^{s}$ were two distinct roots of $\Phi_k$ in $K$, we know because of \cref{eq:reproots2} that they would be the same in $\F$. Therefore, observing expression \cref{eq:discrim}, it follows that $\Delta(\Phi_k \bmod{p})=\Delta(\Phi_k) \bmod{p}=0$, so $p$ divides $\Delta(\Phi_k)$. This is a contradiction with our choice of $p$. Thus, $\zeta^{ps}$ and $\zeta^{s}$ are in fact the same root of $\Phi_k$ in $K$.
	
	Therefore, the equality
	\begin{equation}\label{eq:equalityinzeta}
		\zeta^{ps}=\zeta^{s}
	\end{equation}
	holds in $K$. Writing the above equation in terms of $\theta:=\zeta^{s}$ yields $\theta^{p}=\theta$, where observe that $\theta$ is also a primitive $k$th root of unity. Now, the right-hand side of the equation above does not depend on $p$. The left-hand side only depends on the value of $p\bmod{k}$, since $p$ only appears as an exponent of $\theta$. In conclusion, equality \cref{eq:equalityinzeta} only holds if $p \bmod{k}=1$, that is, if $p\equiv 1\pmod{k}$.
\end{proof}

\begin{remark}\label{remark:trueprimedivphi}
	In the same lines of \cref{remark:trueprimedivf}, \cref{prop:fdivisorsconverse} tells us that $\{p: p\equiv 1\pmod{k}\}\subseteq\Spl_1(\Phi_k)$ (observe there was no need to suppose $\ell\not\equiv 1\pmod{k}$ in the proof of that proposition). This last \cref{prop:fdivisors_l_cong1} tells us that, except for finitely many primes, we have $\Spl_1(\Phi_k)\subseteq\{p: p\equiv 1\pmod{k}\}$. Thus, $\Spl_1(\Phi_k)=\{p: p\equiv 1\pmod{k}\}$, except for finitely many primes.
\end{remark}

Since \cref{prop:fuirreducible} holds except for a finite number of values of the integer $u$, we can further suppose $u$ to be a non-zero multiple of $k$. We then have:
\begin{lemma}\label{lemma:integervalcyclo}
	The equality $f_u(0)=\Phi_k(u)$ holds. Also, every prime divisor of $\Phi_k(u)$ is $\equiv 1 \pmod{k}$. 
\end{lemma}
\begin{proof}
From \cref{eq:polinomi}, one has
\begin{equation}\label{eq:fu0}
	f_u(0)=\prod_{s\in S}(u-\zeta^s)(u-\zeta^{\ell s})=\prod_{a\in G}(u-\zeta^a),
\end{equation}
where we use \cref{eq:GsplitH}. Since the $k$th cyclotomic polynomial is defined by
\begin{equation*}
	\Phi_k(x)=\prod_{a\in G}(x-\zeta^a),
\end{equation*}
it is clear that $f_u(0)=\Phi_k(u)$. Now, since $u$ is a non-zero multiple of $k$, and working mod $k$, we have
\begin{equation}\label{eq:phikuequiv0}
	\Phi_k(u)=\prod_{a\in G}(u-\zeta^a)\equiv(-1)^{\varphi(k)}\prod_{a\in G}\zeta^a=\prod_{a \in G}\zeta^a=1 \pmod{k},
\end{equation}
where we used that Euler's function $\varphi(k)$ is always even for $k>2$. The last equality comes from the fact that the product goes over all $k$th primitive roots of unity: this excludes $-1$ and the roots can be grouped in complex-conjugate pairs\footnote{For this to work, one should make sure that the conjugate of $\zeta^a$ is not itlself. If that was the case, $\zeta^a\overline{\zeta^a}=\zeta^a\zeta^a=\zeta^{2a}=1$ means $2a\equiv 0\pmod{k}$, and since $a$ and $k$ are coprime, $2\equiv 0\pmod{k}$, so $k=1$ or $k=2$. Since we are excluding these two cases, we can group every $k$th primitive root of unity with its distinct complex-conjugate pair.}, so every pair equals one: $\zeta^a\overline{\zeta^a}=\abs{\zeta^a}^2=1$ for every $a\in G$.

We will finally see that $\Phi_k(u)$ is only divisible by primes $\equiv 1 \pmod{k}$. Let $r$ be a prime divisor of $\Phi_k(u)$. From \cref{prop:fdivisors_l_cong1} it follows that $r\equiv 1 \pmod{k}$ or $r$ divides $k$. However, if $r$ divides $k$, then $\Phi_k(u)\equiv 1 \pmod{r}$ because of \cref{eq:phikuequiv0}. This means that $r$ does not divide $\Phi_k(u)$, a contradiction. Therefore, $\Phi_k(u)=f_u(0)$ is only divisible by primes $\equiv 1 \pmod{k}$.	
\end{proof}

 The following theorem uses the Euclidean polynomial $f_u$ to deduce there exist infinitely many primes $\equiv\ell\pmod{k}$ in the case $\ell\not\equiv 1\pmod{k}$.

\begin{theorem}\label{th:infiniteprimes}
There exists an integer $n=n(k,\ell)$ such that if there exists a prime $p \equiv \ell \pmod{k}$ satisfying $p\nmid n$, then there exists a Euclidean proof of the infinitude of primes $\equiv \ell \pmod{k}$.
\end{theorem}
\begin{proof}
	We must first find a special integer $b$ with some special and suitable properties for our Euclidean proof. By hypothesis, pick one prime $p\equiv \ell\pmod{k}$ such that $p$ does not divide $n:=\Delta(f_u)$. Now, by \cref{prop:fdivisorsconverse}, we can find some $b\in\Z$ such that $p$ divides $f_u(b)$. We can be even more precise: $b$ can be chosen so that $p^2$ does not divide $f_u(b)$. If we suppose otherwise ($p^2$ divides $f_u(b)$) let us write a Taylor expansion around $b$:
	\begin{equation} \label{eq:taylor1}
		f_u(b+x)=f_u(b)+f_u'(b)x+\frac{f_u''(b)}{2!}x^2+O(x^3)\in\Z[x].
	\end{equation} 
	In the case $x=p$, \cref{eq:taylor1} reads
	\begin{equation*}
		f_u(b+p)=f_u(b)+f_u'(b)p+\frac{f_u''(b)}{2!}p^2+O(p^3).
	\end{equation*}
	Now, it follows that $f_u(b+p)\equiv f_u(b)+pf_u'(b) \pmod{p^2}$, and since $p^2$ divides $f_u(b)$, we have $f_u(b+p)\equiv pf_u'(b) \pmod{p^2}$. Since $p$ does not divide $\Delta(f_u)$, $f_u$ has no double roots mod $p$, and therefore $f_u'(b)\not\equiv0 \pmod{p}$. This is a direct consequence of \cref{prop:doubleroots}. In all, we have that $f_u(b)\equiv0 \pmod{p^2}$ implies $f_u(b+p)\not\equiv 0 \pmod{p^2}$. In any case, either $f_u(b)$ or $f_u(b+p)$ will not be divisible by $p^2$, but they both are divisible by $p$.
	
	The above remark enables us to build a Euclidean proof using the Euclidean polynomial $f_u$. As Euclid himself did in his famous proof of the infinitude of prime numbers (see \cref{th:Euclid}), to prove that there exist infinitely many primes $\equiv \ell \pmod{k}$ we will proceed by contradiction. 
	
	Suppose there are finitely many primes $\equiv \ell\pmod{k}$ and denote them by $p_1, p_2,\dots,p_m$. Since the prime $p$ in the remark above is $\equiv \ell \pmod{k}$, we can write the list as $p, p_2,p_3,\dots, \allowbreak p_m$ (so $p_1=p$). Now, let $q_1, q_2,\dots,q_t$ be the prime divisors of $\Delta(f_u)$ and let $Q:=q_1q_2\cdots q_tp_2p_3\cdots p_m$. Consider the following congruence equation system:	
	\begin{equation*}
		\begin{cases}      
			c\equiv b\pmod{p^2}\\
			c\equiv 0\pmod{kQ},
		\end{cases}
	\end{equation*}
	where the integer $b$ is the one guaranteed by the remark above. The Chinese Remainder Theorem guarantees the existence of $c\in\Z$ that is a solution to the above system since $p$ does not divide $kQ$. It follows that 
		\begin{equation*}
		\begin{cases}      
			f_u(c)\equiv f_u(b) \pmod{p^2}\\
			f_u(c)\equiv f_u(0) \pmod{kQ}.
		\end{cases}
	\end{equation*}
	In particular, observe that the prime $p$ divides $f_u(c)$, but $p^2$ does not, due to our particular choice of $b$ (recall that we may change $b$ for $b+p$ if necessary).
	
	We will now prove that every prime that divides $f_u(c)$ is $\equiv 1\pmod{k}$ (except for $p$). Let $r$ be a prime divisor of $f_u(c)$ different from $p$. In \cref{th:fdivisors} we have shown that every prime divisor of $f_u$ divides $k$, divides $\Delta(f_u)$, or is $\equiv1, \ell\pmod{k}$. To reach a contradiction, suppose $r\not\equiv 1\pmod{k}$. Thus, $r\equiv \ell\pmod{k}$ or $r$ divides $k$ or $\Delta(f_u)$, so $r$ divides $kq_1q_2\cdots q_tp_2p_3\cdots p_m=kQ$. Since $f_u(c)\equiv f_u(0)\pmod{kQ}$ and $r$ is a divisor of $kQ$, we deduce that $f_u(c)\equiv f_u(0)\pmod{r}$. But $r$ is a divisor of $f_u(c)$, so $f_u(c)\equiv 0 \pmod{r}$. Therefore, $f_u(0)\equiv 0\pmod{r}$. Thus, it must happen that $r$ is a prime factor of $f_u(0)$, all of which are $\equiv 1 \pmod{k}$, thanks to \cref{lemma:integervalcyclo}. This forces $r$ to be $\equiv 1 \pmod{k}$, a contradiction. Therefore, $f_u(c)$ is only divisible by primes $\equiv 1\pmod{k}$ (and by $p$).
	
	Finally, from the fact that $f_u(c)$ has every prime divisor $\equiv 1 \pmod{k}$ except for $p$, it follows, mod $k$, that $f_u(c)=1\cdot1\cdots 1\cdot\ell=\ell$ (note that $\ell$ only appears once because $p\equiv\ell\pmod{k}$ and the fact that $p^2$ does not divide $f_u(c)$). However, observe that $f_u(c)\equiv f_u(0)=\Phi_k(u)\equiv 1\pmod{k}$, due to \cref{lemma:integervalcyclo}. This is a contradiction since $\ell\not\equiv 1 \pmod{k}$. Therefore, the arithmetic progression $\equiv \ell\pmod{k}$ contains infinitely many primes.
\end{proof}
Observe that we have strongly used the equality $f_u(0)=\Phi_k(u)$. It is therefore important to make the following remark.

\begin{remark}
	As we have previously said, the proof of the above theorem is due to Schur, and it is detailed in Murty's article \cite{Murty}. In that article, however, $h_u(z)$ is defined as $h_u(z):=(u-z)(u-z^\ell)$, that is, our definition differs in a sign. Now observe \cref{eq:fu0}. If Murty's definition is followed, then 
		\begin{equation*}
		f_u(0)=\prod_{s\in S}(\zeta^s-u)(u-\zeta^{\ell s})=(-1)^{\abs{S}}\prod_{a\in G}(u-\zeta^a),
	\end{equation*}
	so $f_u(0)=(-1)^{\varphi(k)/2}\Phi_k(u)$. However, $\varphi(k)/2$ is not necessarily even. In order for the equality $f_u(0)=\Phi_k(u)$ to hold, we changed the sign in $h_u(z)$ with respect to Murty's definition.
\end{remark}

One more remark should be made.
\begin{remark}
Observe that to prove the previous theorem we need to suppose the existence of a prime $p\equiv \ell \pmod{k}$ such that $p$ does not divide $\Delta(f_u)$. This hypothesis is indeed necessary. For instance, take $k=15$, $\ell=11$, and $u=15$ (note that $11^2=121\equiv 1 \pmod{15}$). In this case, following \cref{eq:polinomi}, $f_{15}(x)=x^4+884x^3+293206x^2+43243679x+2392743361$, and a quick calculation with SageMath leads to $\Delta(f_{15})=5^3\cdot11^2\cdot19^2\cdot41^2\cdot1091^2$. In this case, both the primes $11$ and $41$ are $\equiv 11\pmod{15}$, but they both divide $\Delta(f_{15})$. 

However, in Murty's article he does not require this additional constraint over $p$. In fact, he states: ``Now pick some prime $p\equiv \ell \pmod{k}$ \emph{so that} $p$ does not divide the discriminant of $f$''. The case $k=15$ and $\ell=11$ is a counterexample to this statement, so the ``so that'' in his article should be changed to a ``such that''. Thus, this extra hypothesis over $p$ is needed for the argument to work\footnote{Proving that there exists at least one such prime for every $k$ and $\ell$ relatively prime is technically demanding. In fact, if there existed an easy proof, Dirichlet \cref{th:Dirichlet} would be easy to establish. See \cref{sec:thhypothesis} in the Appendix for detail.}. 
\end{remark}
	
\subsubsection{Case \texorpdfstring{$\ell\equiv 1\pmod{k}$}{l=1 (mod k)}}\label{sec:Schur_l_cong1}

The particular case when $\ell \equiv 1 \pmod{k}$ can now be easily proved:
	\begin{corollary}\label{corol:infinite_cong1}
		There are infinitely many primes $\equiv 1 \pmod{{k}}$.
	\end{corollary}
	\begin{proof}
		Observe that \cref{prop:fdivisors_l_cong1} tells us that all prime divisors of $\Phi_k$ but finitely many are $\equiv 1\pmod{k}$. However, in \cref{prop:propertiesprimedivisors} we proved that every non-constant polynomial in $\Z[x]$ has infinitely many prime divisors. Since $\Phi_k\in\Z[x]$ is non-constant, the desired result is finally settled.
	\end{proof}

 Therefore, a Euclidean proof for the arithmetic progression $\equiv \ell\pmod{k}$ is available if $\ell^2\equiv 1 \pmod{k}$. In the general case, the proof starts by supposing there are finitely many primes $\equiv \ell \pmod{k}$; it then finds one specific prime $\equiv \ell\pmod{k}$ and finally uses the Euclidean polynomial $f_u$ to reach a contradiction and conclude there are infinitely many primes of this type. In the case $\ell\equiv 1\pmod{k}$, it is enough to characterise the prime divisors of the Euclidean polynomial $\Phi_k$ and observe that this polynomial has infinitely many prime divisors. In both cases, the proofs we developed match our definition of Euclidean proof.

\subsection{The converse problem. Murty Theorem}\label{sec:converseMurty}

We are now interested in showing that $\ell^2\equiv 1\pmod{k}$ is the only case where we can find Euclidean proofs to Dirichlet Theorem in the way defined in \cref{sec:fundamentals}. Thus, let $f\in\Z[x]$ be a monic, irreducible polynomial such that all its prime divisors but finitely many are $\equiv 1, \ell \pmod{k}$, with infinitely many being $\equiv\ell\pmod{k}$. We are then interested in showing that this necessarily implies that $\ell^2\equiv 1 \pmod{k}$. This was first proved by Ram Murty in $1988$. 

To prove his theorem, we will use some algebraic number theory and the Chebotarev Denisty Theorem. We shall follow \cite{Conrad} to prove Murty's claim in greater detail and clarity than that offered in his article \cite{Murty}. Again, let $\zeta$ be a $k$th primitive root of unity, let $K$ be any number field\footnote{Here $K$ denotes a general number field, and not specifically $\Q(\zeta)$, as it did in the previous section.}, and recall that in characteristic zero $K(\zeta)/K$ is a Galois extension since $x^k-1$ is separable over $K$. Hence, $x^k-1$ has $k$ different roots in the splitting field over $K$, which is $K(\zeta)$. Before proceeding to the next theorem, recall the concepts in \cref{sec:ANT}, \cref{sec:notationMurty} and \cref{sec:Chebotarev}. Specifically recall that in \cref{sec:notationMurty} we defined the set $S_1(k,K)$ as
\begin{equation}\label{eq:S1fisrtdef}
	S_1(k,K):=\{b \bmod{k}: p\equiv b \pmod{k} \ \text{for infinitely many $p\in\Spl_1(K)$}\}.
\end{equation}

\begin{theorem}[\textbf{Conrad}]\label{th:conrad}
	Let $\psi:\Gal(K(\zeta)/K)\rightarrow \Gal(\Q(\zeta)/\Q)$, where $\psi(\sigma)=\sigma|_{\Q(\zeta)}$ for every $\sigma\in\Gal(K(\zeta)/K)$. Then, $\im(\psi)=S_1(k,K)$. 
\end{theorem}
\begin{proof}
	 The initial setup is the number field tower of extensions $K(\zeta)/K/\Q$. We will first justify that the set $S_1(k,K)$ can be more conveniently written as
	\begin{equation}\label{eq:seteq}
		S_1(k,K)=\{q \bmod{k}: q\in\Spl_1(K), q \ \text{unramified in $K(\zeta)$}\}.
	\end{equation}
	We will first see that the left side of \cref{eq:seteq} is contained in the right side. Each congruence class in $S_1(k,K)$ contains infinitely many primes $p$ from $\Spl_1(K)$ by definition, and for each of these primes, there exists some prime ideal $\mathfrak{p}$ dividing $p\mathcal{O}_K$ with $f_{K/\Q}(\mathfrak{p}|p)=1$. Now, recall \cref{remark:finiteprimesramify}: in any non-trival number field extension over $\Q$ the number of primes that ramify is finite. Thus, each of the classes in $S_1(k,K)$ has a prime representative in $\Spl_1(K)$, $q$, which is unramified in $K(\zeta)$. 
	
	We will now see that the right side of \cref{eq:seteq} is contained in the left side. Let $q$ belong to $\Spl_1(K)$ with $q$ unramified in $K(\zeta)$. We will prove that $q$ belongs to $S_1(k,K)$ by showing infinitely many primes $p$ lying in $\Spl_1(K)$ that satisfy $p\equiv q\pmod{k}$. By hypothesis, choose $\mathfrak{q}$ dividing $q\mathcal{O}_K$ in $K$ such that $f_{K/\Q}(\mathfrak{q}|q)=1$. 
	
	Since $q$ is unramified in $K(\zeta)$, it follows that $\mathfrak{q}$ is also unramified in $K(\zeta)$. Thus, one can define the Frobenius element $\sigma:=\Frob_\mathfrak{q}$ of $\Gal(K(\zeta)/K)$, which is a unique, well-defined element since $K(\zeta)/K$ is an abelian extension. The defining property of the Frobenius element yields
	\begin{equation}\label{eq:sigmarestr}
		\sigma(\zeta)\equiv\zeta^{\Norm(\mathfrak{q})} \pmod{\mathfrak{b}},
	\end{equation}
	for any prime ideal $\mathfrak{b}$ lying over $\mathfrak{q}$ in $K(\zeta)$. By the canonical isomorphism between $\Gal(\Q(\zeta)/\Q)$ and $(\Z/k\Z)^\times$, we identify $\zeta^{\Norm(\mathfrak{q})}$ with $\Norm(\mathfrak{q}) \bmod{k}$. Therefore, since the restriction $\sigma|_{\Q(\zeta)}$ is fully determined by \cref{eq:sigmarestr} and $\sigma$ fixes $\Q$, we have
	\begin{equation}\label{eq:sigma1}
		\sigma|_{\Q(\zeta)}=\Norm(\mathfrak{q})\bmod{k}=q\bmod{k},
	\end{equation}
	where we use that $\Norm(\mathfrak{q})=q^{f_{K/\Q}(\mathfrak{q}|q)}$. 
	Using Chebotarev Density Theorem for the (abelian) cyclotomic extension $K(\zeta)/K$ (\cref{corol:abelchebotarev}), there exist infinitely many prime ideals $\mathfrak{p}$ in $K$ satisfying 
	\begin{equation}\label{eq:3condcheb}
	\begin{varwidth}{\displaywidth}
		\begin{enumerate}
			\item[(i)] $\mathfrak{p} \ \text{is unramified in} \ K(\zeta)$,
			
			\item[(ii)] $\Frob_\mathfrak{p}=\sigma$,
			
			\item[(iii)] $f_{K/\Q}(\mathfrak{p}|p)=1$,
		\end{enumerate}
	\end{varwidth}
\end{equation}
where $p$ arises from $\mathfrak{p}\cap\Z=p\Z$. In light of \cref{lemma:densityint} and \cref{remark:intinfinite}, the intersection of the set of prime ideals satisfying the third condition and the set of prime ideals satisfying the first two conditions in \cref{eq:3condcheb} is indeed infinite: the density of the set of primes $\mathfrak{p}$ with inertia degree $1$ is equal to one  due to \cref{remark:inertia1density1}, and the density of the unramified primes $\mathfrak{p}$ in $K(\zeta)$ with Frobenius element equal to $\sigma$ is positive due to \cref{corol:abelchebotarev}.

We then have that $p$ belongs to $\Spl_1(K)$ by construction. Therefore, since $\Norm(\mathfrak{p})=p$ and $\sigma=\Frob_\mathfrak{p}$,
\begin{equation*}
	\sigma|_{\Q(\zeta)}=p \bmod{k},
\end{equation*}
which, comparing with \cref{eq:sigma1}, yields $p\equiv q\pmod{k}$ for infinitely many primes $p$ of $\Spl_1(K)$. This finally settles \cref{eq:seteq}. 

We now turn our attention to the main claim in the theorem. Let $H:=\im(\psi)$\footnote{The letter $H$ is not chosen randomly, since we will see that $H:=\im(\psi)=S_1(k,K)$ and a Euclidean polynomial $f$ for the arithmetic progression $\equiv\ell \pmod{k}$ satisfies $S_1(k,f)=\{1, \ell\}=H$, following the notation of \cref{sec:SchurTh}.}. We will start by proving that $S_1(k,K)\subseteq H$. For this goal, pick a congruence class $q \bmod{k}$ in $S_1(k,K)$ in the notation of \cref{eq:seteq}. We know (see \cref{eq:sigma1}) that $q\bmod{k}=\sigma|_{\Q(\zeta)}$, because, by hypothesis, there exists some $\mathfrak{q}$ lying over $q$ in $K$ with $f_{K/\Q}(\mathfrak{q}|q)=1$ and also $q$ is unramified in $K(\zeta)$. This means that $q \bmod{k}$ belongs to $\im(\psi)$ and, hence, $S_1(k,K)\subseteq H$.

We will now prove that $H\subseteq S_1(k,K)$. Let $b \bmod{k}\in \Gal(\Q(\zeta)/\Q)$, with $\sigma|_{\Q(\zeta)}=b\bmod{k}$ for some $\sigma\in\Gal(K(\zeta)/K)$. This way, $b \bmod{k}$ belongs to $H$. Again, using Chebotarev Density Theorem with this automorphism $\sigma$, we have the same three results as in \cref{eq:3condcheb}. In view of this, pick one prime ideal $\mathfrak{p}$ of $K$ so that $p\Z=\mathfrak{p}\cap\Z$ and $p$ belongs to $\Spl_1(K)$. We have that $\Norm(\mathfrak{p})=p$ and $\Frob_\mathfrak{p}=\sigma$. Therefore, we may write 
\begin{equation*}
		\sigma|_{\Q(\zeta)}=\Norm(\mathfrak{p}) \bmod{k}=p\bmod{k}.
\end{equation*}
Thus, $p\equiv b \pmod{k}$, and since the number of such primes $p$ is infinite due to Chebotarev Density Theorem, $b \bmod{k}$ lies in $S_1(k,K)$, using \cref{eq:S1fisrtdef}.
\end{proof}

Now, we have that the subset $S_1(k,K)$ is the image of the morphism $\psi$. Since $\Gal(K(\zeta)/K)$ is a group, one deduces that $S_1(k,K)$ is a subgroup (of $\Gal(\Q(\zeta)/\Q)\cong(\Z/k\Z)^\times$). Murty Theorem now follows easily.
\begin{corollary}[\textbf{Murty}]
	Let $f \in \Z[x]$ be a Euclidean polynomial. Then, $\ell^2 \equiv 1 \pmod{k}$.
\end{corollary}
\begin{proof}	
	Suppose $\ell\not\equiv 1 \pmod{k}$, for otherwise the claim of the corollary is obvious. Assume that we have a Euclidean polynomial, $f$, for the congruence class $\ell \bmod{k}$ with $\gcd(k, \ell)=1$. In other words, we are supposing that $S_1(k, f)=\{1, \ell\}$, for a monic, irreducible polynomial $f$, in light of \cref{rem:equalityS1}. Let $\theta$ be a root of $f$. The same remark tells us that we may identify $S_1(k,\Q(\theta))$ with $S_1(k,f)$.
	
	We can now use \cref{th:conrad} with $K:=\Q(\theta)$. From this result we deduce that $S_1(k,\Q(\theta))=S_1(k,f)=\{1, \ell\}$ is the image of the morphism $\psi$. Thus, $H:=\{1, \ell\}$ is a subgroup of $(\Z/k\Z)^\times$. In particular, $H$ is a group, so, by Lagrange Theorem, the order of $\ell$ must divide the order of $H$, which is $2$. Since we are supposing $\ell\not\equiv 1 \pmod{k}$, the order of $\ell$ must be $2$, so $\ell^2\equiv 1\pmod{k}$.
\end{proof}

Therefore, the complete theorem stands now in its full form: there exists a Euclidean proof of the infinitude of primes $\equiv \ell \pmod{k}$ if and only if $\ell^2\equiv 1 \pmod{k}$\footnote{The condition $\ell^2\equiv 1 \pmod{k}$ is special because it characterises when $H=\{1, \ell\}$ is a group.}. While it is true that we managed to find proofs \textit{à la Euclid} in these cases, the path to obtain them is not elemental. For one part, \cref{th:infiniteprimes} requires a strong hypothesis, since we needed to suppose the existence of one prime $\equiv \ell\pmod{k}$ for every $k$ and $\ell$, which is tantamount to Dirichlet Theorem. We also used Dirichlet's result to show that $f_u$ is a Euclidean polynomial. Moreover, we needed Chebotarev Density Theorem, which is, in fact, a deeper result than Dirichlet Theorem itself.

In \cref{sec:practical} we will see that for \emph{specific} values of $k$ and $\ell$ we are able to find Euclidean \emph{and} elementary proofs of the infinitude of primes $\equiv \ell\pmod{k}$. Once the arithmetic progression is fixed, the existence theorems and hypothesis we needed in this section will be replaced by simple checks.

\subsection{Abundance of Euclidean proofs}\label{sec:sizeEuc}

Now that we know when Euclidean proofs are possible, it is natural to ask how often the condition $\ell^2\equiv 1 \pmod{k}$ occurs. In other words, for a fixed $k$, we ask ourselves how many congruence classes satisfy $\ell^2\equiv 1\pmod{k}$ out of the $\varphi(k)$ classes mod $k$ with infinitely many primes. As a first approach, note that, for every $k$, the congruence class $\ell=k-1$ contains infinitely many primes, and this can be shown in a Euclidean way since $\ell = k-1 \equiv -1 \pmod{k}$, so $\ell^2\equiv(-1)^2=1 \pmod{k}$. The same reasoning shows that a Euclidean proof is available for the class $\ell=1$.

To fully answer our question, recall how the group $G=(\Z/k\Z)^\times$ breaks down into cyclic groups (see \cref{lem:Gbreaks} and \cref{lemma:cyclicgroups} in \cref{sec:extraresults} in the Appendix). In particular,
\begin{equation*}
	(\Z/4\Z)^\times\cong \Z/2\Z, \quad (\Z/8\Z)^\times\cong \Z/2\Z \times \Z/2\Z, \quad (\Z/16\Z)^\times\cong \Z/2\Z \times \Z/4\Z,
\end{equation*}
and every $(\Z/2^r\Z)^\times$ with $r\geqslant 3$ will be isomorphic to a product of two cyclic groups.

In view of \cref{lemma:cyclicgroups} and the Chinese Remainder Theorem, for every $k$, the group $G$ will be a finite product of some cyclic groups (see \cref{eq:cyclicgroups1} and \cref{eq:cyclicgroups2} in the Appendix). Therefore, a residue class in $G$ will be a tuple $\bar{x}:=(x_1, x_2,\dots, x_t)$, where $t$ will be determined by the number of odd divisors of $k$ and by the power of $2$ that divides $k$. If we define the natural number $d_o\geqslant 1$ by $d_o:=\#\{p \ \text{odd prime: $p$ divides $k$}\}$, then the value of $t\geqslant 1$ is given by
	\begin{equation}\label{eq:tdef}
		t=
		\begin{cases}
			d_o, & \text{if $4$ does not divide $k$,}\\
			d_o+1, & \text{if $k\equiv 4\pmod{8}$,}\\
			d_o+2, & \text{if $k\equiv 0 \pmod{8}$,}
		\end{cases}
	\end{equation}
where $2$ precisely dividing $k$ adds no extra terms to the tuple $\bar{x}$,  $2^2=4$ precisely dividing $k$ adds one term to $\bar{x}$, and $2^r$ precisely dividing $k$ for $r\geqslant 3$ adds two terms to $\bar{x}$.

Recall that we are looking for residue classes of order $2$, that is $\bar{x}^2=(x_1^2, x_2^2,\dots, x_t^2)=1$. Thus, every $x_i$, $1\leqslant i \leqslant t$, must have order dividing $2$ (that is, they must be the trivial element or have order $2$). 

Let $p$ be an odd prime and let $r\geqslant 1$ be an integer. The question now turns to finding out how many elements of order $2$ lie in the group $C_{(p-1)p^{r-1}}$ (the cyclic group of order $(p-1)p^{r-1}$). Observe that the order of $C_{(p-1)p^{r-1}}$ is even, since $p$ is odd. Therefore, since $C_n$ has exactly $\varphi(d)$ elements of order $d$ for any $d\in\N$ dividing $n\geqslant 1$, the cyclic group $C_{(p-1)p^{r-1}}$ has exactly $\varphi(2)=1$ element of order $2$, which must be $-1$. For the same reason, $C_{2^{r-2}}$ has $1$ element of order $2$ for every $r\geqslant 3$.

Thus, for $(x_1^2, x_2^2,\dots, x_t^2)=1$ to be true, it must happen that $x_i=\pm 1$, so there are two options for every $x_i$. In conclusion, for a fixed $k$, the number of residue classes that satisfy $\ell^2\equiv 1\pmod{k}$ is $2^t$, with $t$ defined in \cref{eq:tdef}. Therefore, our initial question comes down to the ratio $2^t/\varphi(k)$ for every $k$.

\subsection{An alternative interpretation}\label{sec:interprreciprocity}

 Again denote $K:=\Q(\zeta)$, where $\zeta$ is a $k$th primitive root of unity. Recall that the set $\Spl_1(L)$ is defined for every number field $L$ as
\begin{equation*}
	\Spl_1(L)=\{p: \text{some $\mathfrak{p}$ lying over $p$ in $L$ has $f_{L/\Q}(\mathfrak{p}|p)=1$}\}.
\end{equation*}
A careful reading of the results in \cref{sec:SchurTh} leads to a characterisation of the set $\Spl_1(L)$ in terms of congruences for every subextension $L$ of $K$ satisfying $[K:L]\leqslant 2$.

By the Fundamental Theorem of Galois Theory, every subfield $L$ of $K$ with $[K:L]\leqslant 2$ must be the fixed field of a subgroup $H$ of $\Gal(K/\Q)\cong (\Z/k\Z)^\times=G$ of index at most $2$. The corresponding subset $H\subseteq G$ must be of the form $H=\{1, \ell\}$ for some $\ell\in G$, additionally satisfying $\ell^2\equiv 1\pmod{k}$ for it to be a subgroup of $G$. Observe that every subgroup of index $2$ must be of this form. Observe that in \cref{th:fdivisors} and \cref{prop:fdivisorsconverse} we have effectively studied the prime divisors of a polynomial generating every subfield $L$ of $K$ with $[K:L]=2$. In particular, thanks to \cref{remark:minpoleta}, this polynomial is also the minimal polynomial of the algebraic element $\eta$ which makes $L=\Q(\eta)$. The case $[K:L]=1$ means that $L=K$, and $H$ is the trivial group. We have also studied the prime divisors of the corresponding generating polynomial in \cref{prop:fdivisors_l_cong1}, which in this case is the minimal polynomial of $\zeta$.

Furthermore, in the above-cited results we have described the form of these prime divisors in terms of congruences.  Let $B\subset A$ be two infinite sets, with $\#(A\setminus B)$ being finite. We shall then write $A\simeq B$. With this notation, and recalling \cref{remark:trueprimedivf} and \cref{remark:trueprimedivphi}, we have seen the following:
\begin{enumerate}[label=(\alph*)]
	\item If $[K:L]=2$, the subfield $L=K^H=\Q(\eta)$ is generated by the roots of the irreducible polynomial $h:=f_u$, which is the minimal polynomial of $\eta$. Moreover, the prime divisors of $h$ are
	\begin{equation}\label{eq:SplhL}
		\Spl_1(h)\simeq\{p: p\equiv 1,\ell\pmod{k}\}.
	\end{equation}
	\item If $L=K$, the field $K$ is generated by the roots of the $k$th cyclotomic polynomial, $h:=\Phi_k$, which is the minimal polynomial of $\zeta$. Moreover, the prime divisors of $h$ are
	\begin{equation}\label{eq:SplhK}
		\Spl_1(h)\simeq\{p: p\equiv 1\pmod{k}\}.
	\end{equation}
\end{enumerate}

In order to translate the above reciprocity laws to statements involving the set $\Spl_1(L)$ we need Dedekind Criterion (see \cref{th:criterionfactorization}). Then, $p$ being a prime divisor of $h$ means that $h \bmod{p}$ has a linear factor $\overline{h_i}\in\F_p[x]$, so $\degpol(\overline{h_i})=f_i=1$, where $f_i$ is the inertia degree of the prime ideal corresponding to $\overline{h_i}$ lying above $p$. 

Since the extension $L/\Q$ is Galois, every irreducible factor mod $p$ of $h$ has degree $1$ (see \cref{prop:factGalois}). Dedekind Criterion establishes that the shape of the factorization of $p$ in $\mathcal{O}_L$ mirrors that of $h \bmod{p}$ into irreducible factors. Thus, $p$ has a prime ideal factor $\mathfrak{p}$ with $f(\mathfrak{p}|p)=1$ in $L$ exactly when $p$ belongs to $\Spl_1(h)$. This is effectively a description of the set $\Spl_1(L)$ in terms of $\Spl_1(h)$. Observe that Dedekind Criterion works for the primes in \cref{eq:SplhL} and \cref{eq:SplhK}, since they do not divide $\Delta(\Z[\eta])=\Delta(h)$ by construction.
 
Therefore, one can interpret the results in \cref{sec:SchurTh} as reciprocity laws for $L$ in the following terms:
\begin{enumerate}[label=(\alph*)]
	\item If $[K:L]=2$, then 
	\begin{equation}\label{eq:SplhL1}
	\Spl_1(L)\simeq\{p: p\equiv 1, \ell \pmod{k}\}.
	\end{equation}
	\item If $L=K$, then
	\begin{equation}\label{eq:SplhK1}
	\Spl_1(L)\simeq\{p: p\equiv 1 \pmod{k}\}.
	\end{equation}
\end{enumerate}
The first result is new to the literature, while the second one was already known. Let $g$ be any polynomial generating the extension $L/\Q$ lying below $K$. In general, a characterisation of $\Spl_1(g)$ (and of $\Spl_1(L)$) is well-known if $g$ is a quadratic or cyclotomic polynomial (see \cite{Chebotarev2}). In the first case, the explicit rules to describe $\Spl_1(L)$ arise from the Quadratic Reciprocity Law (see \cref{th:QR}), while the characterisation of $\Spl_1(L)$ we have obtained for $L=K$ is an example of the second case. However, as far as the author is concerned, an explicit characterisation of $\Spl_1(L)$ for $L$ lying below the $k$th cyclotomic field with $[K:L]=2$ had not been explicitly given before.

\begin{remarknonumber}
	While it is true that $\Spl_1(L)$ may contain more primes than those specified in \cref{eq:SplhL1} and \cref{eq:SplhK1}, there can only exist finitely many such primes. Also, the strict equality $\Spl_1(h)=\Spl_1(L)$ is not in general true. Using SageMath one can easily see that there exists some prime divisor of $h$ not $\equiv1,\ell\pmod{k}$, for which every prime ideal $\mathfrak{p}$ lying above $p$ has $f_{L/\Q}(\mathfrak{p}|p)\neq 1$. Thus, $p$ belongs to $\Spl_1(h)$ but does not belong to $\Spl_1(L)$. 

	Take, for example, the case $k=15$ and $\ell=11$. The polynomial generating $L$ is $h(x)=x^4+884x^3+293206x^2+43243679x+2392743361$, with the prime factors of $\Delta(h)$ being $5, 11, 19, 41$ and $1091$. A simple calculation reveals that every prime ideal above the prime $19$ has inertia degree equal to $2$, so $19$ does not belong to $\Spl_1(L)$, but it belongs to $\Spl_1(h)$ (since $19$ divides $h(4)$).
\end{remarknonumber}

\end{document}