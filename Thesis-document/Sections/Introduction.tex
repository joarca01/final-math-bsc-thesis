\documentclass[../main.tex]{subfiles}

\begin{document}

\section{Introduction}\label{sec:intro}
It is a natural question to ask whether there exist infinitely many prime numbers ending in $3$. This can be translated into finding out if the arithmetic progression $10n+3$ for $n\geqslant 0$ contains infinitely many primes. The first terms of this sequence are 
\begin{equation*}
	3, 13, 23, 33, 43, 53, 63, 73, 83,\dots
\end{equation*}
and indeed $3, 13, 23, 43, 53, 73$ and $83$ are primes. Whether the list of such primes is infinite can be determined using Dirichlet Theorem \cite{Dirichlet}, which guarantees that the above progression contains infinitely many primes since $10$ and $3$ are coprime. This theorem only requires the two integers defining the progression, say $k$ and $\ell$, to be relatively prime to ensure the existence of infinitely many primes in $kn+\ell$ for $n\geqslant 0$. The proof of this theorem involves studying advanced properties of $L$-functions and working with Dirichlet characters.

However, proving the weaker claim that there are infinitely many primes is very straightforward, as only basic divisibility properties are needed. Such is the simplicity that Euclid, around the year $300$ BC, already proved this result \cite{Euclid}. One can go a little further and ---using a proof that very much resembles Euclid's idea--- prove that there exist infinitely many primes of the form $6n+5$ \cite{Hardy}. This can also be done with the progression $4n+1$ and $4n+3$ \cite{Lebesgue}. Thus, it is again natural to ask in what progressions $kn+\ell$ a ``Euclidean proof'' can be found to show that infinitely many primes lie in them. 

In $1912$, Issai Schur proved that if $\ell^2\equiv 1 \pmod{k}$, then infinitely many primes could be found following Euclid's idea \cite{Schur}. In $1988$, M. Ram Murty made the term ``Euclidean proof'' precise and further proved that Schur's condition was the only case where a Euclidean proof could be established \cite{Murty}. Therefore, Dirichlet Theorem cannot be fully proved \textit{à la Euclid}. In fact, Schur and Murty show us that Euclidean proofs are rather restricted, being only available when $\ell^2\equiv 1\pmod{k}$. Although many proofs of Dirichlet Theorem only using elemental mathematics exist for specific progressions \cite{GueronTessler, XianzuLin, Selberg, Mestrovic}, in this thesis we are only looking for proofs that mimic Euclid's model.

The objective of this thesis will be to understand every detail that leads to Schur and Murty's theorems. Once these results stand in their full form, we will use their ideas to build our own \emph{automated proof generator}, which will consist of a code that returns a fully Euclidean proof of the infinitude of primes in $kn+\ell$. The values of $k$ and $\ell$ will be supplied by the user, and a proof resembling that of Euclid will be automatically generated, specifically tailored for the given progression. Some cases have already been discussed by several authors. For instance, Paul T. Bateman gives Euclidean proofs for every $\ell$ satisfying $\ell^2\equiv 1\pmod{24}$ \cite{Bateman}. However, giving a complete and systematic method to construct Euclidean proofs for every arithmetic progression satisfying $\ell^2\equiv 1\pmod{k}$ is something new to the literature. Furthermore, we will take advantage of the results we prove in the way to obtain Schur's theorem to characterise some splitting properties of prime numbers in a subfield of the cyclotomic field of degree at most $2$.

This document will be structured as follows. In \cref{sec:fundamentals} we will briefly discuss the basic notions needed to prove the main theorems. In \cref{sec:theoretical} we will prove Schur and Murty's theorems in detail, also quantifying the number of Euclidean proofs available out of every case allowed by Dirichlet Theorem. Furthermore, we will interpret the main results in this section to characterise the splitting of primes. In \cref{sec:practical} we will use the ideas developed in the previous sections to build Euclidean proofs for every possible value of $k$ and $\ell$ satisfying $\ell^2\equiv 1\pmod{k}$. Conclusions and further work will be described in \cref{sec:conclusions}.

\end{document}