\documentclass[../main.tex]{subfiles}

\begin{document}

\section{Conclusions}\label{sec:conclusions}

This thesis admits two possible interpretations. If one is not interested in automated proofs (or even in Euclidean proofs) they could concentrate on reading \cref{sec:SchurTh} and \cref{sec:interprreciprocity} of this thesis. If such is the interest, one will be happy to learn that we have effectively characterised the set $\Spl_1(L)$ in terms of congruences, where $L$ is a field lying under $\Q(\zeta)$ with $[\Q(\zeta):L]\leqslant 2$ ($\zeta$ is a fixed $k$th primitive root of unity). This set contains the primes $p$ that have a prime ideal factor in $L$ whose residue field is $\Z/p\Z$. Our characterisation is effectively a reciprocity law, which is nevertheless hidden in Murty and Schur's method to construct Euclidean proofs. As far as the author is concerned, no such characterisation was explicit in the existing literature. 

The other possible interpretation of this thesis is to see it as a sort of ``instruction manual'' to build Euclidean proofs of the infinitude of primes in the arithmetic progression $kn+\ell$ for $n\geqslant 0$. An intrepid reader may want to embark on understanding why these instructions work, so they will again delve into the details explained in \cref{sec:SchurTh} and \cref{sec:converseMurty}, ultimately understanding why a Euclidean proof is only possible for progressions satisfying $\ell^2\equiv 1 \pmod{k}$. The journey towards this result will require going through Schur and Murty's theorems, and making use of advanced results like Chebotarev Density Theorem. 

A more practical reader will instead pick an arithmetic progression of their choice and use the webpage to learn in an easier, Euclidean way why it contains infinitely many primes (if that is possible for the chosen progression). The final Euclidean proof is indeed simple, since harder existence theorems are replaced with mere checks once $k$ and $\ell$ are fixed. Moreover, the definitive proof makes use of polynomials and certain quantities that have been conveniently chosen for the argument to work nicely. If the reader wants to find out how these quantities are built, going back to \cref{sec:theoretical} is unavoidable, together with a quick review of \cref{sec:practical}.

Regarding this \cref{sec:theoretical}, it is important to note that we have clarified some aspects of Murty's article. For example, we have made clear why the integer $u$ in the polynomial $h_u$ needs to be a non-zero multiple of $k$, which was not previously evident. Moreover, we have also corrected some inaccuracies in that same paper. These include adding a necessary hypothesis to one of its main theorems, as well as defining the polynomial $h_u$ with an opposite sign.

The work we have developed here could be further expanded. For instance, it is natural to ask if there exists a characterisation, in terms of congruences, of $\Spl_1(L)$ in the case $[\Q(\zeta):L]=n$, for $n\geqslant 3$ (and $n$ dividing $\varphi(k)=[\Q(\zeta):\Q]$). As of now, no such characterisation is known. With respect to the automated proof generator, it could be investigated whether an alternative, simpler polynomial $f$ could be used to prove the existence of infinitely many primes $\equiv\ell\pmod{k}$. Indeed, the polynomial $f$ we propose contains large coefficients for big values of $k$, making the final proof somewhat hard to read. Provided a polynomial with smaller coefficients was found, no alternative arguments to bypass the factorization of $f(0)$ or $\Delta(f)$ would be needed.

\end{document}